\documentclass[KVasios-ECE-Dipl.-Thesis.tex]{subfiles} 
\begin{document}
\chapter{Συμπεράσματα -- Επεκτάσεις}

Στο τελευταίο κεφάλαιο συνοψίζουμε τα κύρια σημεία της παρούσης διπλωματικής εργασίας, τα συμπεράσματα που προκύπτουν καθώς και τις πιθανές μελλοντικές κατευθύνσεις--προεκτάσεις. 

\section{Σύνοψη--Συμπεράσματα}

Η παρούσα διπλωματική εργασία πραγματεύεται ένα ιδιαίτερα απαιτητικό θέμα στον κλάδο της ρομποτικής, αυτό του επιδέξιου ρομποτικού χειρισμού με ανθρωπομορφικές διατάξεις ρομποτικών χεριών (με παράλληλες συνεργαζόμενες κινηματικές αλυσίδες).

Αρχικά στη παρούσα εργασία παρουσιάζεται μια αναλυτική βιβλιογραφική επισκόπηση  βασικών θεμάτων που αφορούν τον επιδέξιο ρομποτικό χειρισμό με έμφαση στις χρησιμοποιούμενες τεχνικές ελέγχου. 

Για τη μελέτη και ανάπτυξη λύσεων σε επίπεδο ελέγχου της ρομποτικής λαβής αναπτύξαμε ένα ειδικά προσαρμοσμένο ειδικό περιβάλλον προσομοίωσης εντός του Simulink με τη χρήση των βιβλιοθηκών ανοικτού κώδικα Open Dynamics Engine μεταγλωττισμένων εντός MEX C++S-Function Block με στόχο την δόμηση πλατφόρμας προσομοιώσεων φυσικής πολλών σωμάτων (Multibody Dynamics) με διαχείριση φαινομένων επαφών--συγκρούσεων (Collision Handling). 

Πάνω στη πλατφόρμα αυτή εφαρμόζουμε αρχικά τεχνική ελέγχου επιδέξιας ρομποτικής λαβής σε επίπεδο αντικειμένου βασιζόμενη στη παθητική ιδιότητα ταυτόχρονα με δυναμικό έλεγχο απόσβεσης στο επίπεδο του αντικειμένου. Τα αποτελέσματα αυτών των προσομοιώσεων αρχικά καταδεικνύουν αξιόπιστη εκτέλεση δυναμικών προσομοιώσεων βάσει του API της ODE εντός πάντα του πλαισίου των αυστηρών υποθέσεων που θέσαμε εξ΄ αρχής πάνω σε θέματα κυρίως της φυσικής των επαφών μεταξύ των σωμάτων. Επίσης βάσει των αποτελεσμάτων αυτών αξιολογείται ο συγκεκριμένος αλγόριθμος ο οποίος εν γένει παρουσιάζει ικανοποιητική συμπεριφορά με την κριτική μας να εστιάζει κυρίως στα ακόλουθα σημεία,
\newpage
\begin{itemize}
\item
Μεγάλη εφαπτομενική πίεση σε αντικείμενα κυλινδρικά ή κυβικά από τα ακροδάκτυλα με αποτέλεσμα την πιθανή απώλεια στήριξης λόγου κίνησης εκτός των περιορισμών τριβής του μοντέλου επαφής.
\item
Παρουσία μόνιμων σφαλμάτων τελικής θέσης.
\item
Παρουσία παρασιτικής σύζευξης μεταξύ κινήσεων σε διαφορετικές κατευθύνσεις.
\item
Σφάλμα στη μετρούμενη ακαμψία σε σχέση με αυτή που ορίζουμε σε επίπεδο ελεγκτή. 
\end{itemize}
Τα φαινόμενα αυτά τα αποδίδουμε στη τεχνική ελέγχου εσωτερικών δυνάμεων μέσω ορισμού συντελεστών ακαμψίας μεταξύ ακροδακτύλων και εικονικού πλαισίου αντικειμένου. Οι δυνάμεις που δημιουργούνται από αυτές τις συνδέσεις δεν απεικονίζονται στο μηδενικό χώρο της μήτρας λαβής με αποτέλεσμα να δημιουργούν μία συνισταμένη εξωτερική δύναμη στο αντικείμενο υπό χειρισμό  η οποία αποτελεί και αιτία των προβλημάτων που αναφέραμε.

Πάνω σε αυτό το θέμα προτείνουμε λύση βασιζόμενη στη γεωμετρία της επιφάνειας του αντικειμένου και συγκεκριμένα την προβολή στον εσωτερικό χώρο της μήτρας λαβής δυνάμεων κάθετων στην επιφάνεια της επαφής. Θεωρούμε ότι αυτό αποτελεί ρεαλιστική προσέγγιση καθώς η λήψη του κάθετου διανύσματος στην επιφάνεια είναι πιθανή μέσω συστημάτων όρασης--αφής τα οποία, στις περισσότερες των περιπτώσεων, είναι παρόντα στις ρομποτικές εφαρμογές με στόχο τον επιδέξιο χειρισμό. Με αυτό το τρόπο επιτυγχάνουμε τη διατήρηση των δυνάμεων των ακροδακτύλων εντός των περιορισμών τριβής του μοντέλου της επαφής αποφεύγοντας πιθανές ολισθήσεις και ταυτόχρονα απόλυτο έλεγχο των εσωτερικών δυνάμεων. Οι πραγματοποιηθείσες προσομοιώσεις επαληθεύουν ακριβώς αυτές τις υποθέσεις δίνοντας μας μικρότερα μόνιμα σφάλματα θέσης, μικρότερες συζεύξεις μεταξύ διαφορετικών κατευθύνσεων κίνησης καθώς και συνεπή μήτρα ακαμψίας σε σχέση με αυτή που ορίζουμε.

Τέλος εισάγουμε την βαρυτική επίδραση στο σύστημα η οποία όντας μία εξωτερική δύναμη δημιουργεί μόνιμα σφάλματα θέσης. Για τα δάκτυλα έχουμε εισάγει ήδη δυναμικό βαρυτικό όρο αλλά όχι και για το αντικείμενο. Επεκτείνοντας τις ακαμψίες ως προς την μετατόπιση--περιστροφή εισάγουμε όρο αντιστάθμισης βαρύτητας σε επίπεδο αντικειμένου, γεωμετρικά συνεπή, με επιτυχή αποτελέσματα.\\

\newpage
\section{Επεκτάσεις -- Μελλοντικές κατευθύνσεις}

\textbf{Προσομοίωση}
\\

Η προσομοίωση δυναμικών συστημάτων αποτελεί ένα εξαιρετικά μεγάλο και κρίσιμο κομμάτι της μηχανικής των συστημάτων για την κατασκευή ολοκληρωμένων επιτυχών εφαρμογών. Όπως ήδη αναφέραμε η Open Dynamics Engine αποτελεί μία ιδιαίτερα δημοφιλής λύση για την προσομοίωση συστημάτων ρομποτικής χάρη κυρίως, στην ευρωστία της, την σταθερότητά της, την ευελιξία της καθώς και την ταχύτητάς της. Παρ΄ όλα αυτά παρουσιάζει σημαντικές ελλείψεις και απλοποιήσεις περιοριζόμενη παράλληλα σε προσομοιώσεις απόλυτα στερεών σωμάτων, ενώ αντιθέτως παρατηρούμε ότι ένα μεγάλο μέρος της ρομποτικής έρευνας προσανατολίζεται στην ρομποτική εύκαμπτων--μαλακών στοιχείων (Soft -- Robotics). Αν και μπορεί κανείς για αυτές τις περιπτώσεις μπορεί κανείς να αναζητήσει στη βιομηχανία προσαρμοσμένες εμπορικές εφαρμογές το πρόβλημα της ρεαλιστική δυναμικής προσομοίωσης συστημάτων πολλαπλών σωμάτων με ταυτόχρονη διαχείριση φαινομένων συγκρούσεων παραμένει ακόμα ανοικτό.

Έτσι λοιπόν, σήμερα, θα μπορούσε κανείς να αναπτύξει ένα νέο πακέτο λογισμικού ανοικτού κώδικα για τη δυναμική προσομοίωση προσαρμοσμένο στις σημερινές ανάγκες της ρομποτικής κοινότητας. Ενδιαφέρον έχει ότι ενώ έχουν γίνει πολυποίκιλες προσπάθειες για τη δημιουργία πλατφόρμας ανάπτυξης ρομποτικών συστημάτων (Webots, V-Reptile, ROS κ.α), όλες βασίζονται σε απλουστευμένες μηχανικές φυσικής προσανατολισμένες κυρίως στην ανάπτυξη εφαρμογών ψυχαγωγίας και γραφιστικής με κύριες τις ODE, Bullet και PhysX όπου αντικειμενικός τους στόχος είναι η ταχεία και αληθοφανής απόκριση και όχι η δυναμική προσομοίωση με στόχο την απόλυτη ακρίβεια των αποτελεσμάτων.\\




\textbf{Έλεγχος}
\\

Στο κομμάτι του ελέγχου αρχικά κανείς μπορεί να παρατηρήσει ότι στο σύστημά μας κάνουμε υπόθεση τέλειου ελεγκτή ροπής στις αρθρώσεις όπου το σήμα μας μεταφράζεται απόλυτα στην τελικώς ασκούμενη ροπή στις αρθρώσεις Αυτό φυσικά αφαιρεί αρκετά σημαντικό κομμάτι από την αξιοπιστία της προσομοίωσης. Έτσι λοιπόν μία προφανής προσθήκη θα ήταν η δημιουργία υποσυστήματος για το χαμηλό επίπεδο του ελεγκτή ροπής το οποίο θα περιλαμβάνει τη δυναμική του συστήματος επενέργησης.

Ως προς το κομμάτι του συνολικού δυναμικού συστήματος του μοντέλου της ρομποτικής λαβής, όπως ήδη έχουμε αναφέρει, στην ανάλυση μας έγιναν κάποιες απλουστευτικές παραδοχές προς διευκόλυνση του προσδιορισμού του δυναμικού όρου απόσβεσης. Αυτό έχει σαν αποτέλεσμα μία μη αμελητέα ασυνέπεια ως προς το χαρακτήρα των μεταβατικών χαρακτηριστικών για διαφορετικές θέσεις -- τοποθετήσεις του συστήματος. Θα ήταν χρήσιμο να διερευνηθεί προς αυτή τη κατεύθυνση ένα ακόμη πιο ολοκληρωμένο σχήμα ελέγχου το οποίο θα λαμβάνει υπ' όψη του και τους όρους Coriolis  -- γινομένων γωνιακών ταχυτήτων τους οποίους παραλείψαμε στην ανάλυσή μας.

Στη συνέχεια αυτό που γίνεται προφανές είναι ότι στον αλγόριθμό μας λαμβάνουμε ως δεδομένη τη γνώση για τα αδρανειακά χαρακτηριστικά του αντικειμένου υπό χειρισμό κάτι το οποίο υποσκάπτει τη προσπάθεια για κατασκευή ρομποτικών συστημάτων με αυτοματοποιημένη δράση σε μη δομημένο περιβάλλον. Είναι δυνατή η εισαγωγή σχήματος προσαρμοστικού ελέγχου ο οποίος θα δύναται να ρυθμίσει τις παραμέτρους του συστήματος δυναμικά για διαφορετικά αδρανειακά χαρακτηριστικά αντικειμένων. 

Αν και το σύστημα ικανοποιεί το χαρακτηρισμό ``επιδέξιο", στη πραγματικότητα υπάρχουν σημαντικά περιθώρια βελτίωσης ως προς αυτό, αναλογιζόμενοι ότι τα περιθώρια προσανατολισμού του αντικειμένου είναι στη πραγματικότητα περιορισμένα.  Εξαιρετικά ενδιαφέρουσα προοπτική, θα ήταν λοιπόν, η επέκταση του σχήματος για πραγματοποίηση finger gaiting ή και κατάλληλης ολίσθησης -- κύλισης στα ακροδάκτυλα, για την αύξηση της επιδεξιότητας, εισάγοντας ουσιαστικά με αυτό τον τρόπο δυναμική διακριτών φαινομένων.

Μία εργασία με επίσης μεγάλο ενδιαφέρον, θα ήταν η δοκιμή των συγκεκριμένων αλγορίθμων σε συστήματα με διαφορετικά μοντέλα επαφών που περιλαμβάνουν και φαινόμενα 2ης τάξης.\\


\textbf{Πραγματικό Σύστημα}
\\

Η θεωρητική ανάλυση και προσομοίωση αποτελούν αναπόσπαστα και αναγκαία κομμάτια κάθε προσπάθειας ανάπτυξης συστημάτων αυτόματου ελέγχου. Όλα αυτά φυσικά μικρό νόημα αποκτούν αν δεν υπάρχει εφαρμογή στον πραγματικό κόσμο και σε πραγματικές συνθήκες.




\end{document}
