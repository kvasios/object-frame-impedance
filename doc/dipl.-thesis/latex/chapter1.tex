\documentclass[KVasios-ECE-Dipl.-Thesis.tex]{subfiles} 
\begin{document}
\chapter{Εισαγωγή}
Στο παρόν εισαγωγικό κεφάλαιο επιχειρούμε αρχικά μία προσέγγιση διατύπωσης του ορισμού της ρομποτικής επιστήμης και των ρομποτικών συστημάτων αντίστοιχα. Στη συνέχεια  αναφέρουμε κάποια στοιχεία προχωρώντας σε μία βασική ανάλυση για τη σημερινή προσφορά αλλά και τους στόχους εξέλιξης της ρομποτικής. Φυσική σύνδεση στη μελέτη σχεδόν κάθε ρομποτικού συστήματος, ιδιαίτερα αυτών με στόχο την αυτόνομη, προσαρμοστική--ευφυή δράση, αποτελούν τα βιολογικά συστήματα και ιδιαίτερα ο ίδιος ο άνθρωπος. Αμέσως μετά εστιάζουμε την προσοχή μας στο κομμάτι εκείνο της ρομποτικής το οποίο ασχολείται με τον επιδέξιο χειρισμό, που είναι και το αντικείμενο της παρούσης διπλωματικής εργασίας, αναφέροντας κάποια βασικά χαρακτηριστικά των μέχρι τώρα προσπαθειών για τη κατασκευή τέτοιων διατάξεων. Τέλος, παρουσιάζουμε τη δομή της εργασίας ανά κεφάλαιο με τις αντίστοιχες περιλήψεις αυτών.
%\addcontentsline{toc}{chapter}{Εισαγωγή}
%Μελέτη της επίδρασης διαφόρων τεχνικών βελτιστοποίησης κώδικα που στοχεύουν στην αξιοποίηση της cache
%\apendix 
\section{Ρομποτική Επιστήμη}
%\addcontentsline{toc}{section}{Ρομποτική Επιστήμη}
%\textbf{Ρομποτική}\\

Ένας κομψός, γενικός και αφαιρετικός ορισμός της ρομποτικής επιστήμης είναι ο ακόλουθος.
\begin{quotation}
``Ρομποτική ορίζεται ως η επιστήμη που μελετά την ευφυή σχέση μεταξύ αντίληψης και δράσης \cite{Siciliano:2008fk}” 
\end{quotation}
 
%\textbf{Ρομπότ}\\

Η λέξη ρομπότ προέρχεται από τη τσέχικη λέξη \emph{robota} η οποία χρησιμοποιείται για να περιγράψει τη βαρετή ή και κοπιαστική, πιθανά εξαναγκαστική, δουλειά.\\

Για τα ρομποτικά συστήματα έχουν προταθεί διάφοροι ορισμοί. Παρουσιάζουμε κάποιους από τους πιο χαρακτηριστικούς.
\begin{quotation}
“Ρομπότ ορίζεται ως μηχανικός ή εικονικός πράκτορας, συνήθως σύνθετη ηλεκτρομηχανική συσκευή, η οποία οδηγείται από προγραμματιζόμενα στοιχεία”\\
\vspace{-1ex}
\hfill \textit{Wikipedia}
\end{quotation} 
\vspace{1ex}

\begin{quotation}
“Μηχανή ικανή να πραγματοποιήσει μία πολύπλοκη σειρά δράσεων αυτόματα, προγραμματιζόμενη από υπολογιστικά στοιχεία" 

“Μηχανή ανθρωπόμορφη, ικανή να αντιγράψει κάποιες συγκεκριμένες ανθρώπινες κινήσεις και ικανότητες πραγματοποιώντας αυτές με αυτόματο τρόπο"

“Αναφέρεται σε πρόσωπο το οποίο συμπεριφέρεται μηχανικά με μη συναισθηματικό τρόπο"\\
\vspace{-1ex}
\hfill \textit{Google Dictionary}
\end{quotation} 
\vspace{1ex}

\begin{quotation}
“Μηχανή, ορισμένες φορές ανθρωπόμορφη, ικανή να πραγματοποιήσει μία γκάμα από συχνά πολύπλοκες ανθρώπινες εργασίες βάσει εντολών ή πρότερου προγραμματισμού αυτής"

“Μηχανή ή συσκευή η οποία δρα αυτόματα ή μέσω απομακρυσμένου χειρισμού"\\
\vspace{-1ex}
\hfill \textit{American Heritage Dictionary}
\end{quotation} 
\vspace{1ex}

\begin{quotation}
“Αυτοματοποιημένη μηχανή προγραμματισμένη για την εκτέλεση ειδικών μηχανικών εργασιών αυτόματα ή υπό τη καθοδήγηση φυσικού προσώπου"\\
\vspace{-1ex}
\hfill \textit{Collins English Dictionary}
\end{quotation} 
\vspace{1ex}

Στην περίπτωση του εικονικού πράκτορα (virtual agent) ο συνηθέστερος όρος που χρησιμοποιείται είναι αυτός του \emph{bot}.\\

Επιχειρούμε μία αφαιρετική σύνθεση των ανωτέρω ορισμών με σκοπό τη συστημική προσέγγιση.
\begin{quotation}
Η επιστήμη της ρομποτικής ασχολείται με τη μελέτη συστημάτων τα οποία συνδυάζοντάς αισθητηριακά στοιχεία, υπολογιστική ικανότητα -- νοημοσύνη και επενεργητές -- κινηματικές διατάξεις είναι σε θέση να προσλάβουν πληροφορία από το περιβάλλον, να την επεξεργαστούν και τελικά να επενεργήσουν σε αυτό με τρόπο που εξυπηρετεί το σκοπό για τον οποίο σχεδιάστηκαν.    
\end{quotation} 

Το πεδίο της ρομποτικής επιστήμης εμπεριέχει ένα εξαιρετικά ευρύ φάσμα επιμέρους επιστημονικών πεδίων με πιο χαρακτηριστικά αυτά της επιστήμης της ηλεκτρονικής, της επιστήμης υπολογιστών, της γνωσιακής επιστήμης (κυρίως με το κομμάτι της τεχνητής νοημοσύνης), της μηχατρονικής, νανοτεχνολογίας, εμβιομηχανικής κ.α. Η επιτυχής ολοκλήρωση αυτών για την σύνθεση μίας επιτυχημένης ρομποτικής εφαρμογής αποτελεί ιδιαίτερη πρόκληση.

Η ευρύτητα του πεδίου των εφαρμογών, τα πολλαπλά επιστημονικά πεδία αλλά και το νεαρό ίσως της ρομποτικής επιστήμης αποτελούν πιθανότατα κάποιους από τους λόγους που δικαιολογούν την απουσία ενός αυστηρού ορισμού, οικουμενικά αποδεκτού.

Μέχρι σήμερα οι περισσότερο εκτεταμένες, και πιο επιτυχημένες, εφαρμογές ρομποτικής περιορίζονται στις βιομηχανικές γραμμές παραγωγής επιτελώντας επαναλαμβανόμενες εργασίες σε αυστηρά δομημένο και προβλέψιμο περιβάλλον. Θεωρείται μάλιστα ότι οι σχετικές τεχνολογίες των βιομηχανικών ρομπότ έχουν φτάσει σε ένα ώριμο σημείο \cite{HirzingerBalsOtterStelter2005}. Στις περισσότερες από αυτές τις εφαρμογές γίνεται χρήση αποκλειστικά τεχνικών ελέγχου θέσης, δίνοντας με αυτό το τρόπο ένα σαφές στίγμα για την τάση της βιομηχανίας να προτιμά παραδοσιακές, κλασικές τεχνικές ελέγχου. 

Το στοίχημα όμως για τη ρομποτική ήταν και είναι η επιτυχής διείσδυση στον πραγματικό κόσμο με συστήματα ικανά για ευφυή αυτόνομη δράση όπου το περιβάλλον είναι μη δομημένο και κυριαρχεί η εμφάνιση απρόβλεπτων γεγονότων. Μια τέτοια εξέλιξη θα είχε και πολύ σημαντικά οφέλη και στον τομέα της βιομηχανίας όπου δρουν ρομποτικά συστήματα, καθώς υπολογίζεται ότι το κόστος διαμόρφωσης του χώρου εργασίας του ρομποτικού συστήματος είναι τέσσερις φορές μεγαλύτερο από αυτό της εγκατάστασης του ρομπότ αυτού καθ' εαυτού. 

Μία τέτοια εξέλιξη προϋποθέτει αυξημένες ικανότητες αντίληψης, γνωσιακές ικανότητες καθώς και ικανότητες κινηματικής και δυναμικής επενέργησης στον περιβάλλοντα χώρο. Έχουν γίνει πολλές προσπάθειες για ανάπτυξη τέτοιων συστημάτων από κυβερνητικούς αλλά και ιδιωτικούς φορείς παγκοσμίως, κυρίως από τις προηγμένες χώρες, παρ’ όλα αυτά σε καμία περίπτωση δεν αποτελούν κομμάτι της καθημερινότητας των κοινωνιών στο βαθμό που θα αναμενόταν. Πολύ χαρακτηριστικό παράδειγμα είναι αυτό του πυρηνικού ατυχήματος στη Fukushima της Ιαπωνίας, όπου ενώ αναμενόταν μία ιδιαίτερα εξελιγμένη χώρα στο τομέα της ρομποτικής να διαθέτει ρομποτικά συστήματα για αντιμετώπιση καταστροφών, τελικά ήταν εργάτες αυτοί που εκτέθηκαν στο επικίνδυνο αυτό περιβάλλον για τον έλεγχο της κρίσης. Αυτή η υστέρηση δημιουργεί έντονα ερωτήματα σχετικά με το ποιους δρόμους και κατευθύνσεις πρέπει να ακολουθήσει η ρομποτική στα επόμενα χρόνια. 

Εξαιρετικά μεγάλο ενδιαφέρον πάνω σε αυτά τα θέματα παρουσιάζει ο καινούριος Διαγωνισμός Ρομποτικής της Υπηρεσίας Προηγμένων Ερευνών Συστημάτων Άμυνας των Ηνωμένων Πολιτειών (DARPA Robotics Challenge), ο οποίος θέτει σαν πρόκληση τη κατασκευή ανδροειδών ρομποτικών συστημάτων ικανά για δράση σε σενάρια διάσωσης σε επικίνδυνα, υποβαθμισμένα ανθρωπογενή περιβάλλοντα. Αν λάβουμε υπ' όψη την επιτυχία του προηγούμενου αντίστοιχου διαγωνισμού της DARPA για τη κατασκευή αυτό--οδηγούμενων οχημάτων (στη πολιτεία της Nevada ήδη εκδόθηκε η πρώτη άδεια κυκλοφορίας για το αυτο -- οδηγούμενο όχημα της Google), τότε οι εξελίξεις για τα αυτόνομα ρομποτικά συστήματα αναμένεται να είναι ραγδαίες. 

Τα τελευταία χρόνια διαμορφώνονται κάποιες τάσεις οι οποίες ενισχύουν τη τάση ως προς την εξέλιξη προηγμένων συστημάτων ρομποτικής ικανά για αυτόνομη δράση στο πραγματικό κόσμο. 
Τα ευφυή κινητά τηλέφωνα (smartphones), με τη τρομακτική ανάπτυξη που σημείωσαν τα τελευταία χρόνια έριξαν δραματικά το  κόστος των επιμέρους μικροσυστημάτων που ενσωματώνουν, παρέχοντας φθηνές και αξιόπιστες λύσεις σε ολοκληρωμένα κυκλώματα για την παροχή άπλετης υπολογιστικής ισχύος αλλά και μεγάλου όγκου δεδομένων από μεγάλη γκάμα από αισθητήρες όπως, γυροσκόπια, επιταχυνσιόμετρα, αισθητήρες εγγύτητας, αφής, ορατού ηλεκτρομαγνητικού φάσματος, ανίχνευσης γήινου μαγνητικού πεδίου κ.α δίνοντας έτσι δυνατότητες σε νέα κλίμακα για την ανάπτυξη ρομποτικών συστημάτων\cite{drones}. 

Ταυτόχρονα, η γνωσιακή επιστήμη (cognitive science) έχει αρχίσει να επιδεικνύει τα τελευταία χρόνια σημαντική απτή πρόοδο με ιδιαίτερη ανάπτυξη στο κομμάτι της τεχνητής νοημοσύνης με την ανάπτυξη εντυπωσιακών εφαρμογών (βλ. IBM Blue Gene, Watson, Google Car--Stanford SUV, MS Kinect κ.α). Παρ’ όλη αυτή την εντυπωσιακή ανάπτυξη του τομέα της τεχνητής νοημοσύνης πρέπει να σημειώσουμε ότι ακόμα αυτά τα συστήματα απέχουν πολύ από να χαρακτηριστούν πραγματικά ευφυή βάσει των αντίστοιχων βιολογικών προτύπων. Σήμερα είναι κοινά αποδεκτό στην επιστημονική κοινότητα ότι το κομμάτι της τεχνητής νοημοσύνης αποτελεί ουσιαστικά τη τροχοπέδη στην ανάπτυξη ολοκληρωμένων αυτόνομων ρομποτικών συστημάτων \cite{Yoshikawa2010}. 

Σημαντική πρόοδο συναντά κανείς και στο τομέα των υλικών με τη κατασκευή αποδοτικών συνδέσμων, επενεργητών καθώς και ικανών μαλακών--εύκαμπτων στοιχείων (soft robotics). 
%Στο συγκεκριμένο τομέα μάλιστα θεωρείται ότι τα υλικά έχουν φτάσει σε ένα ώριμο σημείο προσφέροντας δυνατότητες ικανές ώστε να συγκριθούν ακόμα και με βιολογικά πρότυπα\cite{Yoshikawa2010}. 

Συνδυάζοντας αυτά τα στοιχεία πολλοί ήταν εκείνοι που ανέμεναν μία έκρηξη καινοτόμου δραστηριότητας στο τομέα αυτό κατά τη δεκαετία που διανύουμε, παρόμοια με αυτή της πληροφορικής τη δεκαετία του '90, επενδύοντας σε σχετικές start--up εταιρείες με στόχο τη πλατιά διεύρυνση της αγοράς των ρομποτικών συστημάτων με αποτέλεσμα στη Sillicon Valley αυτή τη στιγμή να δρουν πάνω από 80 σχετικές εταιρίες με προβλέψεις για ακόμα μεγαλύτερη ανάπτυξη στο προσεχές μέλλον \cite{svrobo,ieee_silicon_valey_robotics}.

Στο σχήμα \ref{fig:Robots_Collage} παρουσιάζονται ενδεικτικά κάποια από τα πιο εντυπωσιακά παραδείγματα πρόσφατων ρομποτικών συστημάτων τα οποία και σκιαγραφούν τις τάσεις που είναι και πολύ πιθανό να οδηγήσουν τις μελλοντικές εξελίξεις.


\begin{figure}[htbp]%  figure placement: here, top, bottom, or page
   		\centering
		%   \includegraphics[scale=0.6]{images_kefalaio3/denavit.pdf} 
   		\resizebox{0.8\textwidth}{!}{\input{images_kefalaio1/Robots/Robots_Collage.tex}}
   		\caption{Ευφυή Ρομποτικά Συστήματα: Ξεκινώντας από αριστερά προς τα δεξιά και ανά γραμμή, AlphaDog (DARPA), Atlas (DARPA), Tdedy--One (WASEDA University Sugano Laboratory TWENDY team), Seagle (FESTO), Asimo (Honda), Opportunity (NASA), PR2 (Willow Garage), Robonaut (NASA), Industrial Arm (KUKA), UAV Drone, DaVinci,Google Autonomous Car}
   		\label{fig:Robots_Collage}
\end{figure}
\phantom{1}
%\textbf{Αλληλεπίδραση}
%\\
%
%Από τον ορισμό του ρομποτικού συστήματος προκύπτει ότι η επιτελούμενη εργασία εκτελείται μέσω κάποιας μορφής αλληλεπιδράσεως με το περιβάλλον και τα στοιχεία αυτού. Αυτή η αλληλεπίδραση μπορεί να είναι αμφίδρομη ή όχι,  μπορεί έχει πολλές εκφάνσεις μορφές και φυσικούς δρόμους. Για παράδειγμα ένα ρομποτικό σύστημα όπου ο αντικειμενικός σκοπός κατασκευής του αποτελεί την εξερεύνηση -παρακολούθηση λαμβάνει απλά πληροφορία από το περιβάλλον μέσω ηλεκτρομαγνητικής αλληλεπίδρασης (συστήματα όρασης) ή απ‘ ευθείας ανταλλαγής δυνάμεων (αισθητήρες δύναμης-αφής, haptic system) μη έχοντας σκοπό να το μεταβάλλει. Αντίθετα ένα ρομποτικό σύστημα κατασκευής-συναρμολόγησης ή εκτέλεσης οικιακών εργασιών, ενσωματώνει κατάλληλα στοιχεία με σκοπό την μεταβολή του ευρύτερου περιβάλλοντος και των στοιχείων του(manipulation).
%
%Εστιάζοντας στα ρομποτικά συστήματα τα οποία επιτελούν εργασίες μέσω κατάλληλων κινηματικών μηχανικών διατάξεων, στο κομμάτι που αφορά την αλληλεπίδραση μέσω δυνάμεων, γίνεται προφανές και για τις δύο αυτές περιπτώσεις, ότι απαιτείται κάποιο προσαρμοσμένο, ειδικά σχεδιασμένο, τελικό στοιχείο δράσης. Στις βιομηχανικές εφαρμογές για παράδειγμα, συναντά κανείς τον ίδιο τύπο κινηματικής δομής και ανάλογα με την επιτελούμενη εργασία προσαρμόζεται στο άκρο αυτής το κατάλληλο εργαλείο. Αυτή η λύση έχει προφανή πλεονεκτήματα αυτά της απλότητας, της απόλυτης αποτελεσματικότητας στην εκτέλεση της εργασίας καθώς το εργαλείο είναι ακριβώς αυτό που χρειάζεται, οικονομίας κλίμακας στη κατασκευή των ρομποτικών στοιχείων κα.  Από την άλλη πλευρά βασικό μειονέκτημα αποτελεί το γεγονός ότι για την εγκατάσταση των συστημάτων αυτών απαιτείται και γενική προσαρμογή του περιβάλλοντος με αυστηρά δομημένο τρόπο καθιστώντας τη γενική διάταξη άκαμπτη σε μεταβολές (το κόστος της διαμόρφωσης του περιβάλλοντος χώρου είναι 4 φορές υψηλότερο από το κόστος εγκατάστασης του ρομπότ αυτού καθ' αυτού).   \\
\newpage
\textbf{Συνεισφορά Ρομποτικών Συστημάτων}
\\

Οι γενικοί τομείς δραστηριότητας των ρομποτικών συστημάτων είναι κυρίως,
\begin{itemize}
\item
Βιομηχανικές εφαρμογές, κυρίως σε διαδικασίες παραγωγής στη βαριά βιομηχανία.
\item
Αεροδιαστημική
\item
Ιατρική, με ιδιαίτερη έμφαση στο κομμάτι της ρομποτικής χειρουργικής.
\item
Προσθετική, με κατασκευή τεχνητών μελών καθώς και διαδικασίες αποκατάστασης ασθενών.
\item
Βοήθεια ηλικιωμένων.
\item
Δράση σε επικίνδυνα--υποβαθμισμένα περιβάλλοντα.
\item
Αυτοματοποίηση καθημερινών εργασιών σε χώρους εργασίας και κατοικίας.
\item
Μη επανδρωμένος πόλεμος.
\item
Εφαρμογές ψυχαγωγίας και διασκέδασης.
\end{itemize}
\phantom{4}

\textbf{Βιολογικό Πρότυπο}
\\

Τα βιολογικά συστήματα αποτελούν τα πιο επιτυχημένα παραδείγματα τέτοιων συστημάτων, από τους πιο ταπεινούς μικροοργανισμούς μέχρι τα εξελιγμένα θηλαστικά. Η κλασματική (fractal) δομή--μορφολογία και η προκύπτουσα δυναμική αντίστοιχα αυτών των συστημάτων δεδομένης μίας διαδικασίας βιολογικής εξέλιξης εκατοντάδων εκατομμυρίων ετών η οποία οδηγεί σε βελτιστοποίηση της ενεργειακής διαχείρισης μέσω των μεταβολικών διαδικασιών μας δίνει αντίστοιχα και μία εικόνα της πολυπλοκότητας του μη δομημένου φυσικού περιβάλλοντος καθώς και των προκλήσεων της ρομποτικής επιστήμης ως προς τη κατασκευή ευφυών αυτόνομων, ημιαυτόνομων, συστημάτων. Βασικό μοντέλο και πηγή έμπνευσης στο όλο εγχείρημα αποτελεί φυσικά ο ίδιος ο άνθρωπος.

Κάνοντας μία απλουστευτική συστημική προσέγγιση σχέσεως εισόδου - εξόδου στο “σύστημα άνθρωπος” προκύπτουν τα ακόλουθα ενδιαφέροντα στοιχεία που αποκαλύπτουν σε ένα μικρό μόνο βαθμό τη πολυπλοκότητα αυτού.

Προσεγγιστικά για τις ανθρώπινες νευρικές αισθητηριακές απολήξεις έχουμε συνολικά 300.000.000 αισθητηριακές εισόδους -- νευρικές απολήξεις εκ των οποίων
	120.000.000 Ράβδοι και 6.000.000 Κώνοι στον αμφιβληστροειδή χιτώνα κάθε ματιού.
	40.000.000 νευρικές απολήξεις για την όσφρηση.
	3.500.000 νευρικές απολήξεις για την αφή.
	15.000-20.000 Ακουστικοί νευρικοί υποδοχείς σε κάθε ωτό,
	και 10.000 υποδοχείς γεύσης. 


Η έξοδος του συστήματος ουσιαστικά εκφράζεται μέσω του μυοσκελετικού συστήματος το οποίο και επενεργεί στο περιβάλλον. 270 οστά και 650 μυικά στελέχη (μέχρι 850 ανάλογα με τον τρόπο που γίνεται η καταμέτρηση) αναλαμβάνουν να φέρουν εις πέρας το ιδιαίτερα πολύπλοκο παιχνίδι νευτώνειας δυναμικής στο πλαίσιο της αλληλεπιδράσεως με το περιβάλλον. 

Μπορούμε να πούμε για το σύστημα συνολικά ότι λαμβάνει 300.000.000 εισόδους και έχει μόνο 800 εξόδους με κυρίαρχο αισθητηριακό σύνολο αυτό της όρασης \cite{sensory_aparatus}.

Ανάλογα συνοψίζεται και το πρόβλημα του σχεδιασμού κατάλληλου ευφυούς ρομποτικού ελεγκτή βάσει της υπάρχουσας τεχνολογίας με την έκφραση “pixels to torques”.

\begin{figure}[htbp]%  figure placement: here, top, bottom, or page
   		\centering
		%   \includegraphics[scale=0.6]{images_kefalaio3/denavit.pdf} 
   		\resizebox{0.8\textwidth}{!}{%\documentclass{article}
%
%\usepackage{tikz}
%\usetikzlibrary{arrows}
%\usepackage{verbatim}
%
%\begin{document}
%\pagestyle{empty}

\setlength\fboxsep{0pt}
\setlength\fboxrule{1pt}

\tikzstyle{int}=[draw, fill=blue!20, minimum size=2em]
\tikzstyle{init} = [pin edge={to-,thin,black}]

\begin{tikzpicture}[node distance=2.5cm,auto,>=latex']
\node [int] (image) {\fbox{\includegraphics[width=0.25\textwidth]{images_kefalaio1/DaVinci_Man.pdf}}};
\node (input) [left of=image,node distance=5 cm, coordinate] {};
 \node [coordinate] (end) [right of=image, node distance=5cm]{};
\path[->] (input) edge node {$u \in \mathbb{R}^{\sim3 \cdot10^8}$} (image);
\path[->] (image) edge node {$y \in \mathbb{R}^{ 600\sim850}$} (end) ;
    
    
%    \node [int, pin={[init]above:$v_0$}] (a) {$\frac{1}{s}$};
%    \node (b) [left of=a,node distance=2cm, coordinate] {a};
%    \node [int, pin={[init]above:$p_0$}] (c) [right of=a] {$\frac{1}{s}$};
%    \node [coordinate] (end) [right of=c, node distance=2cm]{};
%    \path[->] (b) edge node {$a$} (a);
%    \path[->] (a) edge node {$v$} (c);
%    \draw[->] (c) edge node {$p$} (end) ;

 %    \draw[black,ultra thick] (0,0) rectangle (\textwidth,6.2);
\end{tikzpicture}

%\end{document}}
   		\caption{Τάξη μεγέθους για τον αριθμό εισόδων--εξόδων στο σύστημα άνθρωπος}
   		\label{fig:DaVincis-Human-System}
\end{figure}


\section{Επιδέξιος Ρομποτικός Χειρισμός}
Η κατασκευή ρομποτικών χεριών αποτέλεσε από τις σημαντικότερες περιοχές έρευνας από την αρχή της ρομποτικής επιστήμης. Αυτό είναι λογικό, καθώς ο χειρισμός των στοιχείων του περιβάλλοντος μέσω της απ’ ευθείας ανταλλαγής δυνάμεων αποτελεί από τις θεμελιωδώς επιδιωκόμενες ρομποτικές λειτουργίες αποτελώντας και μία από τις βασικότερες, αναπόσπαστες προϋποθέσεις για τη δράση, ανδροειδών κυρίως, ρομποτικών συστημάτων σε μη δομημένα περιβάλλοντα.

Στη προσπάθεια αυτή είναι αδύνατο να παραβλέψουμε το αντίστοιχο βιολογικό πρότυπο το οποίο δεν είναι άλλο από το ίδιο το ανθρώπινο χέρι. Η αποτελεσματικότητα του βιολογικού προτύπου γίνεται άμεσα αντιληπτή μέσα από την, προφανή, παρατήρηση ότι το απόλυτο σύνολο της ανθρωπογενούς δραστηριότητας αποτελεί αποτέλεσμα της δράσεως χεριού -- νου. Από ανθρωπολογικής πλευράς αποδεικνύεται ότι είναι η μηχανική επιδεξιότητα, αυτή καθ΄αυτή, μία από τις βασικές αιτίες που πυροδότησαν την ανάπτυξη του ανθρώπινου νου \cite{Bicchi:2000fk}.
Η επιδεξιότητα του ανθρώπινου χεριού βρίσκεται ακόμα και σήμερα αρκετά πιο μπροστά από οποιαδήποτε αντίστοιχη μηχανική κατασκευή και πιθανότατα θα κρατήσει αυτή τη πρωτοκαθεδρία για πολύ ακόμα. Ενώ σε επιμέρους ικανότητες και τεχνικά χαρακτηριστικά, όπως ταχύτητα και ανθεκτικότητα, κάποια ρομποτικά χέρια φαίνεται να υπερτερούν, είναι το εύρος των ικανοτήτων του ανθρώπινου χεριού να αντιμετωπίζει με απόλυτη επιτυχία μία εντυπωσιακά μεγάλη γκάμα εφαρμογών που το καθιστούν ουσιαστικά σχεδιαστικό πρότυπο. Η απάντηση πάντως στο κατά πόσο πρέπει ο σχεδιαστής να επιδιώκει είτε τον ανθρωπομορφισμό είτε κάποιο βέλτιστο σχεδιασμό ως προς συγκεκριμένες παραμέτρους εξαρτάται από την εκάστοτε εφαρμογή και τις απαιτήσεις αυτής \cite{Bicchi:2000fk}. 

Ιδιαίτερα χαρακτηριστικό παράδειγμα αυτού είναι η τεχνολογία αρπάγης του ρομποτικού συστήματος στη πλάτη του διαστημικού λεωφορείου της NASA στο οποίο ενώ αρχικά είχε προταθεί ένα κλασσικό σχήμα αντικριστών κινηματικών αλυσίδων, κατά τα ανθρωπομορφικά πρότυπα, τελικά προτιμήθηκε μία λύση η οποία αρπάζει τα αντικείμενα στο διαστημικό χώρο μέσω ενός ανοιγοκλειόμενου διαφράγματος \cite{latching_end_effector_NASA}. Μία ακόμα ενδιαφέρουσα περίπτωση εναλλακτικής πρότασης χειρισμού κάνει χρήση ενός σφαιρικού ελαστικού γομώδη σάκου γεμάτου με οργανικό υλικό. Ο σάκος αυτός εφόσον έρθει σε επαφή με κάποιο αντικείμενο λαμβάνει συμμορφωτικά το σχήμα του αντικειμένου και στη συνέχεια με διαδικασία αναρρόφησης ο σάκος συμπιέζεται με αποτέλεσμα να αγκαλιάζει τελικά σε απόλυτο βαθμό το αντικείμενο εξασφαλίζοντας πολύ καλή μηχανική σύνδεση \cite{Brown02112010}.


Ένα βασικό στοιχείο ανθρωπομορφισμού στο σχεδιασμό που εφαρμόζεται όλο και περισσότερο στα επιδέξια ρομποτικά χέρια είναι η χρήση μαλακών υλικών στα άκρα με στόχο την ενίσχυση μίας ήπιας συμπεριφοράς μηχανικής συμμόρφωσης (compliant behavior), υψηλού εύρους ζώνης, η οποία καθιστά τη διάταξη πιο ικανή για εύρωστη λαβή και χειρισμό \cite{LottiTiezziVassuraBiagiottiMelchiorri2004} συγκριτικά με τα απολύτως άκαμπτα στοιχεία. Τροχοπέδη στο πεδίο αυτό αποτελεί η έλλειψη αυστηρού μαθηματικού φορμαλισμού περιγραφής της δυναμικής συμπεριφοράς των εύκαμπτων ρομποτικών συστημάτων \cite{NguyenArimoto2002}, πράγμα που αποτελεί και το βασικότερο λόγο για τον οποίο οι όποιες τεχνικές ελέγχου προς αυτό το πεδίο αποτελούν προέκταση των κλασσικών για τα άκαμπτα ρομποτικά χέρια \cite{KhalilPayeur2011}.  

%Ανάλυση ανθρώπινου χεριού μερικά στοιχεία ανατομίας.
%
%
%
%
%Ρομποτική λαβή, προσπάθειες μέχρι τώρα.


%Ακολουθούν κάποιες από τις βασικές προσπάθειες της επιστημονικής κοινότητας στη κατασκευή ρομποτικών χεριών παρατηρώντας παράλληλα πως διαμορφώνονται οι κυρίαρχες τάσεις στις μέχρι τώρα προσεγγίσεις.

\begin{figure}[htbp]%  figure placement: here, top, bottom, or page
   		\centering
		%   \includegraphics[scale=0.6]{images_kefalaio3/denavit.pdf} 
   		\resizebox{0.8\textwidth}{!}{\input{images_kefalaio1/Hands/Hands_Collage.tex}}
   		\caption{Επιδέξια Ανθρωπομορφικά Ρομποτικά Χέρια: Ξεκινώντας από αριστερά προς τα δεξιά και ανά γραμμή: DLR Hand (German Aerospace Center),DLR Hand 2 (German Aerospace Center), Twedy--one Hand (WASEDA University Sugano Laboratory TWENDY team), Robonaut Hand Schematic (NASA), Shadow Robot Hand, FESTO ExoHand}
   		\label{fig:DaVincis-Human-System}
\end{figure}


Ένα βασικό στοιχείο που παίζει καθοριστικό ρόλο στο σχεδιασμό είναι το κατά πόσο το ρομποτικό χέρι προορίζεται να προσαρμοστεί σε μία ήδη υπάρχουσα ρομποτική κινηματική διάταξη βραχίωνα, στη θέση του τελικού στοιχείου δράσης ή αν αποτελεί κομμάτι ενός πλήρως προσαρμοσμένου μηχανικού σχεδίου χεριού--μπράτσου. Στη πρώτη περίπτωση το σύνολο της μηχανικής διάταξης (επενεργητές, στοιχεία μετάδοσης της κίνησης) τοποθετείται είτε εντός του χεριού είτε σε ειδικό κλωβό τοποθετημένο πάνω εξωτερικά και πάνω από τη παλάμη, χώρος ο οποίος δεν αποτελεί χώρο εργασίας του συστήματος. Στη περίπτωση πλήρως προσαρμοσμένου σχεδίου δίνεται σαφώς μεγαλύτερη ελευθερία ως προς τη μηχανολογική διάταξη. Σε συστήματα υψηλών επιδόσεων προτιμάται το ανθρωπομορφικό πρότυπο με τοποθέτηση των επενεργητών στο σύνδεσμο ανάμεσα από καρπό και αγκώνα και μετάδοση της κίνησης μέσω τενόντων. Αυτή η διάταξη τοποθετεί το μεγαλύτερο μέρος των μηχανικών στοιχείων--άρα και το μεγαλύτερο μέρος της μάζας της διάταξης--κοντά στο μπράτσο και στον κύριο κορμό με θετικές επιπτώσεις στα συνολικά δυναμικά χαρακτηριστικά όλου του συστήματος αλλά και του χεριού αυτού καθ' αυτού διαθέτοντας τελικά ελαφρύτερους συνδέσμους--αρθρώσεις.\\


%
%Insert pictures of the robothands!!! 
%\\
%
%\textbf{Soft Robotics}
%\\

\newpage
\section{Οργάνωση Κειμένου}
\begin{itemize}
\item Στο \textit{κεφάλαιο 2} γίνεται μία βιβλιογραφική επισκόπηση των τεχνικών ελέγχου επιδέξιων ρομποτικών λαβών. Αρχικά το συνολικό πρόβλημα του χειρισμού αποδομείται στα επιμέρους υποπροβλήματα που το απαρτίζουν. Στη συνέχεια δίνονται βασικοί ορισμοί εισάγοντας ένα πρώτο στάδιο φορμαλισμού έτσι όπως έχει αναπτυχθεί για το θέμα της ρομποτικής λαβής στη σχετική βιβλιογραφία. Αμέσως μετά ακολουθούν οι τρόποι περιγραφής του συστήματος της ρομποτικής λαβής μέσω της μαθηματικής μοντελοποίησης και των αντίστοιχων τεχνικών αυτής καθώς και η περιγραφή της αισθητηριακής σύνθεσης που εφαρμόζεται σε τέτοιες διατάξεις. Εφόσον λοιπόν έχει δοθεί περιγραφή για τη μοντελοποίηση αλλά και τα αισθητηριακά στοιχεία, παρουσιάζονται κάποια κυρίαρχα σχήματα τεχνικών ελέγχου για συστήματα ρομποτικής λαβής. 
\item Στο \textit{κεφάλαιο 3} αναλύουμε το θεωρητικό υπόβαθρο πάνω στο οποίο στηρίζεται και η υλοποίησή στο πλαίσιο της παρούσης εργασίας. Επιχειρείται μία ιεραρχικά δομημένη προσέγγιση η οποία ξεκινάει από την αναλυτική περιγραφή του κινηματικού -- δυναμικού μοντέλου του ρομποτικού δακτύλου και κατ' επέκταση χεριού, κατά τα κινηματικά -- δυναμικά πρότυπα του ρομποτικού χεριού DLR Hand 2. Αναλύονται κάποιες θεμελιώδεις ιδιότητες του δυναμικού μοντέλου βάσει της θεωρίας παθητικών συστημάτων, στις οποίες στηρίζεται και ο νόμος ελέγχου. Δομείται η στατική ανάλυση της λαβής και ο ορισμός της μήτρας λαβής καθώς και η αντίστοιχη λύση του προβλήματος των ελαχίστων τετραγώνων. Εισάγεται ο ορισμός του  εικονικού πλαισίου του ρομποτικού χεριού και στη συνέχεια βάσει αυτού δομούνται οι αντίστοιχες συναρτήσεις δυναμικού οι οποίες ορίζουν την ακαμψία σε επίπεδο αντικειμένου και στα 6 καρτεσιανά επίπεδα κίνησης καθώς και στον εσωτερικό χώρο δυνάμεων. Στη συνέχεια παρουσιάζεται η δυναμική τεχνική αντιμετώπισης των πλεονασματικών βαθμών ελευθερίας του ρομποτικού καθώς και ο σχεδιασμός του όρου απόσβεσης. 
 Εντοπίζονται τα βασικά προβλήματα -- μειονεκτήματα του ελέγχου στο χώρο των εσωτερικών δυνάμεων μέσω του ορισμού ακαμψίας ως προς το εικονικό πλαίσιο, και προτείνεται τεχνική βασιζόμενη στα γεωμετρικά χαρακτηριστικά του αντικειμένου. 
 Τέλος προτείνεται και αναλύεται κατάλληλη προέκταση των όρων γραμμικής και περιστροφικής ακαμψίας με στόχο την βαρυτική αντιστάθμιση.
\item Στο \textit{κεφάλαιο 4} περιγράφεται αναλυτικά η μεθοδολογία που εφαρμόσθηκε για την υλοποίηση, σε δομημένη πλατφόρμα προσομοίωσης, των τεχνικών ελέγχου που μελετώνται στο πλαίσιο της παρούσης εργασίας, και παρουσιάζονται τα αποτελέσματα των δοκιμών που εκπονήθηκαν. Δίνονται κάποια βασικά στοιχεία για τα επιμέρους δομικά στοιχεία της υλοποίησης τα οποία είναι, το μοντέλο--πρότυπο ρομποτικό χέρι DLR Hand 2, το λογισμικό -- API δυναμικών προσομοιώσεων ODE το οποίο ενσωματώνεται σε περιβάλλον Simulink MEX C++ S--Function Block. Ακολουθούν τα αποτελέσματα της προσομοίωσης τα οποία περιλαμβάνουν τη βηματική απόκριση του συστήματος για στροφική και μεταφορική κίνηση καθώς και τη μέτρηση της ακαμψίας μέσω της άσκησης δυνάμεων--ροπών στο αντικείμενο υπό χειρισμό. Οι μετρήσεις αυτές πραγματοποιούνται αρχικά, για τον απλό αλγόριθμο εγγενούς παθητικού ελέγχου (IPC) και στη συνέχεια επαναλαμβάνονται για το σχήμα ελέγχου των εσωτερικών δυνάμεων βάσει γεωμετρίας αντικειμένου (IPC -- IF), προκειμένου να εξαχθούν τα κατάλληλα συμπεράσματα για τις συγκριτικές επιδόσεις αυτών. Τα αποτελέσματα καταδεικνύουν την αποτελεσματικότητα της προτεινόμενης τεχνικής.
\item Το \textit{κεφάλαιο 5} αποτελεί τον επίλογο της εργασίας, όπου γίνεται μία σύνοψη, γενική εξαγωγή συμπερασμάτων και μία διερεύνηση πιθανών μελλοντικών προεκτάσεων.
\end{itemize}



\end{document}
