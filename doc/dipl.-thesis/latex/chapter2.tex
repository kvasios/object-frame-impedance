\documentclass[KVasios-ECE-Dipl.-Thesis.tex]{subfiles} 
\begin{document}
%\chapter{Επιδέξια Ρομποτική Λαβή \\ Τεχνικές Ελέγχου}
\chapter{Έλεγχος Επιδέξιας Ρομποτικής Λαβής: Επισκόπιση}
Στο παρόν κεφάλαιο επιχειρούμε μία γενική βιβλιογραφική επισκόπηση κυρίως επί των τεχνικών ελέγχου της επιδέξιας ρομποτικής λαβής αλλά και ορισμένων επιμέρους θεμάτων. \\

\textbf{Αποδόμηση Προβλήματος}
\vspace{1ex}

Προκειμένου να αναλυθεί το σύνθετο θέμα του επιδέξιου χειρισμού δύναται να αποδομηθεί και να εξεταστεί ξεχωριστά βάσει των επιμέρους ημι--αυτόνομων ακόλουθων διαδοχικών εργασιών. 
\begin{enumerate}
\item Προσέγγιση του αντικειμένου--επιφάνειας στόχου,  φάση κατά την οποία δεν υπάρχει περιορισμός στη κίνηση και γίνεται κατάλληλος σχεδιασμός τροχιάς προσέγγισης των σημείων επαφής. 
\item Πραγματοποίηση επαφής και άρα συγκεκριμένος περιορισμός στη κίνηση μέσω ανταλλαγής δυνάμεων. Σε αυτή τη φάση επιθυμούμε τον έλεγχο ως προς τις δυνάμεις--ροπές αλληλεπίδρασης αλλά και ως προς τη θέση του τελικού στοιχείου δράσης.  
\item Διαδικασία χειρισμού. Η διαδικασία χειρισμού επιτυγχάνεται ελέγχοντας ταυτόχρονα τις δυνάμεις αλληλεπίδρασης καθώς και την αντίστοιχη θέση των σημείων επαφής στο χώρο.
\end{enumerate}

Αυτά τα επιμέρους προβλήματα συνδέονται μεταξύ τους καθώς για παράδειγμα η επιλογή των σημείων επαφής επηρεάζει σε μεγάλο βαθμό τη μετέπειτα ικανότητα χειρισμού. Παρ’ όλα αυτά μπορούν να εξεταστούν και ανεξάρτητα αναπτύσσοντας τεχνικές ελέγχου για κάθε ένα από αυτά τα υποπροβλήματα ξεχωριστά. \\

Στη παρούσα διπλωματική εργασία ασχολούμαστε κυρίως με το τελευταίο κομμάτι αυτό του επιδέξιου χειρισμού του αντικειμένου εφ’ όσον έχει πραγματοποιηθεί επιτυχής προσέγγιση των σημείων επαφής.\\

Για την εξέλιξη της ανάλυσής αλλά τη πραγματοποίηση οποιασδήποτε άλλης περιγραφής δίνουμε αρχικά κάποιους θεμελιώδεις κλασικούς ορισμούς σχετικά με τα χαρακτηριστικά και τις ιδιότητες του συστήματος της λαβής. 

%\addcontentsline{toc}{chapter}{Εισαγωγή}
%Μελέτη της επίδρασης διαφόρων τεχνικών βελτιστοποίησης κώδικα που στοχεύουν στην αξιοποίηση της cache
%\apendix 
\section{Κλασικοί Αναγκαίοι Ορισμοί}
%\addcontentsline{toc}{section}{Ρομποτική Επιστήμη}

\textbf{Επιδεξιότητα}
\vspace{1ex}

Αναφέρεται στην ικανότητα αλλαγής θέσης και προσανατολισμού του αντικειμένου υπό χειρισμό από μία αρχική διάταξη αναφοράς στο χώρο σε μία άλλη τυχαία ορισμένη μέσα στο χώρο εργασίας των δακτύλων~\cite{KhalilPayeur2011,Bicchi:2000fk}.\\

Αποτελεί μία αρκετά ευρεία έννοια η οποία αφορά την ταυτόχρονη ικανότητα αλλά και σταθερότητα στην πραγματοποίηση κινήσεων του χειριζόμενου αντικειμένου από τη παλάμη και τα δάκτυλα.
\\

\textbf{Ευρωστία Λαβής}
\vspace{1ex}

Η ικανότητα να διατηρείται σταθερό το αντικείμενο ανεξάρτητα διαταραχών οποιασδήποτε μορφής (όπως για παράδειγμα μη αναμενόμενες δυνάμεις, λάθος εκτιμήσεις των χαρακτηριστικών του αντικειμένου) ενώ ταυτόχρονα οι εσωτερικές δυνάμεις λαβής (internal/grip forces) περιορίζονται έτσι ώστε να μη προκληθούν φθορές στο γενικό σύστημα.
\\

\textbf{Χειρισμός υπό Ανθρώπινη Παρουσία\phantom{1}(Human Operability)}
\vspace{1ex}

Η δυνατότητα για εύκολη και ασφαλή αλληλεπίδραση σε περιβάλλον με ανθρώπινη φυσική παρουσία.\\

Σε πρακτικές εφαρμογές αυτά τα βασικά κριτήρια μπορεί να είναι δύσκολο να συνυπάρξουν απαιτώντας από το σχεδιαστή τη λήψη αποφάσεων συμβιβασμού~\cite{Bicchi:2000fk}.\\

\textbf{Κλειστότητα Λαβής (Grasp Closure) -- Μοντελοποίηση Επαφής}
\vspace{1ex}


Μία λαβή καλείται κλειστή, αν και μόνο αν, βρίσκεται σε κατάσταση ισορροπίας για οποιοδήποτε τυχαίο διάνυσμα γενικευμένων εξωτερικών δυνάμεων το οποίο δρα πάνω στο αντικείμενο~\cite{BicchiKumar2000}.\\

Οι δυνάμεις εξισορρόπησης που αντιστέκονται στη μετακίνηση του αντικειμένου παράγονται διά μέσου της άμεσης επαφής της ρομποτικής διάταξης με το αντικείμενο. Η κρισιμότητα της επιλογής των κατάλληλων σημείων επαφής αλλά και η δυναμική αυτών, είναι καθοριστική για τη συμπεριφορά του όλου συστήματος.   
Η μοντελοποίηση της επαφής είναι εξαιρετικά σημαντική στην ανάλυση του συστήματος της ρομποτικής λαβής--αντικειμένου. Γενικά στη βιβλιογραφία συνήθως γίνονται απλουστευτικές παραδοχές στη μοντελοποίηση των επαφών θεωρώντας τες σημειακές με τριβή βάσει του μοντέλου Coulomb. Μία βασική κλασική κατηγοριοποίηση ως προς τα μοντέλα των επαφών είναι η ακόλουθη.

\begin{itemize}
\item \textbf{Σημειακή Επαφή με ή χωρίς τριβή -- \textit{Hard Finger}}. Ασκεί δύναμη ως προς την κατεύθυνση του αντικειμένου (τοπικά κάθετα στην επιφάνεια του αντικειμένου στα σημεία επαφής) και στην περίπτωση ύπαρξης τριβής και εφαπτομενικά.
\item \textbf{Εύκαμπτη επαφή -- \textit{Soft Finger}}. Μπορεί να ασκήσει επιπλέον και ροπή εξ' επαφής ως προς τον κάθετο άξονα στο σημείο επαφής.
\end{itemize}
 
 
 Πολύ σημαντικές ιδιότητες που λαμβάνονται υπ’ όψη στο μοντέλο της επαφής είναι η ιξώδης-ελαστική (visco-elastic behaviour, rigid, isotropically elastic) συμπεριφορά της, οι συνθήκες κύλισης και ολίσθησης, δηλαδή οι στατικοί και δυναμικοί όροι του μοντέλου, καθώς και το κατά πόσο τα σώματα που βρίσκοναι σε επαφή παρουσιάζουν κύλιση (\textit{rolling contact}) ή ολίσθηση (\textit{sliding contact}). Πάνω σε αυτά τα θέματα κινούνται οι κλασικές εργασίες του Salisbury και Mason, καθηγητών ερευνητών του MIT, οι οποίοι με τις κλασικές πλέον εργασίες τους τη δεκαετία του 80΄ έθεσαν τα θεμέλια στην ανάλυση και το σχεδιασμό των επιδέξιων ρομποτικών λαβών. Ο Salisbury έδειξε πρώτος ότι ο μικρότερος θεωρητικά αριθμός βαθμών ελευθερίας αναγκαίος για την επίτευξη επιδεξιότητας σε χέρι με άκαμπτα δάκτυλα, χωρίς φαινόμενα κύλισης και ολίσθησης είναι 9, με το κάθε δάκτυλο να διαθέτει τουλάχιστον 3 DOFs. Αντίστοιχα προέβη και στο σχεδιασμό--κατασκευή ρομποτικού χεριού με αυτά τα χαρακτηριστικά.
 
Για την αύξηση της ευελιξίας ως προς τη λειτουργία του χειρισμού αρκετοί ερευνητές εισήγαγαν πλεονάζοντες βαθμούς ελευθερίας στα σχέδιά τους με πιο κλασική προσθήκη αυτή του ενός επιπλέον μεν, συζευγμένου δε, βαθμού ελευθερίας στο κάθε δάκτυλο στην άπω διαφαλαγγική άρθρωση (distal interphalangeal joint). Αυτή η προσέγγιση μιμείται το ανθρώπινο πρότυπο και ουσιαστικά οι βαθμοί ελευθερίας παραμένουν 3. 

Μία βασική προϋπόθεση για τη διείσδυση συστημάτων επιδέξιου χειρισμού στο πραγματικό κόσμο αποτελεί η μείωση της πολυπλοκότητας σε κάθε επίπεδο της υλοποίησης~\cite{Bicchi:2000fk}. Η ιδιαίτερα πολύπλοκη φύση των επιμέρους μηχανικών μερών έρχεται σε απ’ ευθείας σύγκρουση με σχεδιαστικά κριτήρια τα οποία επιβάλουν υψηλό βαθμό αξιοπιστίας, μικρό κόστος, μικρό βάρος. Η μηχανική πολυπλοκότητα της κατασκευής του ρομποτικού χεριού αποτυπώνεται χαρακτηριστικά στον αριθμό των χρησιμοποιούμενων επενεργητών ο οποίος ξεκινά από 9 και μπορεί να φτάσει τους 32 ή και παραπάνω.

Πάνω στο θέμα της βελτιστοποιημένης σχεδίασης είναι πολύ σημαντικό να σημειώσουμε ότι ο απαιτούμενος αριθμός βαθμών ελευθερίας για την επίτευξη επιδεξιότητας είναι απόλυτα συνδεδεμένος από τις αρχικές παραδοχές που κάνουμε πάνω στο μοντέλο των επαφών. Για παράδειγμα υπό τη παραδοχή ότι οι επαφές είναι "εύκαμπτες" soft-finger οι ελάχιστοι βαθμοί ελευθερίας για κάθε δάκτυλο για την επίτευξη επιδεξιότητας προκύπτουν 4~\cite{Bicchi:2000fk}.

Έτσι λοιπόν μπορούν να οριστούν πιο απλές διατάξεις από το μηχανολογικό επίπεδο μέχρι το επίπεδο του ελέγχου οι οποίες θα μπορούν να είναι εξίσου επιδέξιες στο χειρισμό εκμεταλλευόμενες εναλλακτικές τεχνικές όπως αυτές των, \textit{Regrasping \& Finger Gaiting} ή και των τεχνικών που εκμεταλλεύονται φαινόμενα κύλισης και ολίσθησης των επαφών,  \textit{Rolling \& Sliding}~\cite{Bicchi:2000fk}. Για αυτές τις τεχνικές μεγάλη τροχοπέδη αποτελεί η δυσκολία καθορισμού κλειστής μαθηματικής περιγραφής για αυτά τα φαινόμενα σε βαθμό που θα μπορούσε να είναι πραγματοποιήσιμη η εφαρμογή κατάλληλης τεχνικής ελέγχου. Ιδιαίτερα για το φαινόμενο της κύλισης μεταξύ εύκαμπτων μικροδάκτυλου και άκαμπτου αντικειμένου έχει ορισθεί το Lagrangian δυναμικό μοντέλο του συστήματος μόνο για δύο δάκτυλα 2 DOFs στις 2 διαστάσεις~\cite{NguyenArimoto2002}. Για τις τεχνικές αυτές θα αναφερθούμε πιο αναλυτικά στην επόμενη ενότητα.\\
 
Βάσει αυτής της μοντελοποίησης των επαφών όπως την αναλύσαμε παραπάνω ορίζονται τα αντίστοιχα είδη της κλειστότητας της λαβής που μπορούν να προκύψουν.\\
 
\textbf{Κλειστότητα ως προς τη μορφή  -- Form Closure}
\vspace{1ex}

Αναφέρεται στην ικανότητα της λαβής να αποτρέψει κινήσεις του αντικειμένου βασιζόμενη σε περιορισμούς που δημιουργούνται από επαφές μονομερείς, χωρίς τριβή~\cite{Bicchi:2000fk,BicchiKumar2000,KhalilPayeur2011}. \\

Το πρόβλημα αυτό, άμεσα συσχετιζόμενο και με το σχεδιασμό μηχανικών συσκευών ακινητοποίησης αντικειμένων στο χώρο για διαδικασίες συναρμολόγησης και κατασκευής, μελετάται από τον 19ο αιώνα με τα πρώτα σημαντικά θεωρητικά αποτελέσματα από τον πατέρα της κινηματικής των μηχανικών συστημάτων, όπως χαρακτηριστικά αποκαλείται ο Franz Reuleaux. Η θεωρητική μελέτη δείχνει ότι χρειάζονται τουλάχιστον 4 επαφές χωρίς τριβή για την ακινητοποίηση ενός αντικειμένου στο επίπεδο και 7 για τον 3Δ χώρο. Το πρόβλημα της κλειστότητας ως προς τη μορφή παρουσιάζεται και με την ανάστροφη μορφή, αυτή της ανάλυσης, όπου δεδομένης της λαβής εξετάζεται αν υπάρχουν διαθέσιμοι βαθμοί ελευθερίας για το αντικείμενο, και αν ναι προς ποια κατεύθυνση. Μία επέκταση του κλασικού ορισμού περί κλειστότητας ως προς τη μορφή, υπό τον όρο “immobilization problem”, λαμβάνει υπόψη φαινόμενα 2ης τάξης, τα οποία αναπτύσσονται λόγω της σχετικής μορφολογίας της επιφάνειας των δύο σωμάτων,  παρέχοντας τελικά μεγαλύτερη ακρίβεια στην ανάλυση~\cite{Bicchi:2000fk}.\\

\textbf{Κλειστότητα ως προς τη δύναμη -- Force Closure}
\vspace{1ex}

Η κλειστότητα ως προς δύναμη προσδιορίζει την ικανότητα της λαβής να αντιστέκεται σε οποιεσδήποτε εξωτερικές δυνάμεις και συνήθως αναφέρεται σε σημειακές επαφές με τριβή. Στη τελευταία περίπτωση μάλιστα η λαβή μπορεί να αντεπεξέλθει οποιασδήποτε δύναμης ή ασκούμενης ροπής δεδομένης της ύπαρξης αρκετά μεγάλης κάθετης δύναμης στο σημείο της επαφής~\cite{Bicchi:2000fk,BicchiKumar2000,KhalilPayeur2011}.\\

Επίσης οι λαβές οι οποίες είναι κλειστές ως προς δύναμη, ανάλογα με το κατά πόσο τα σημεία περιορισμού της κίνησης του αντικειμένου είναι ενεργά στοιχεία (δηλαδή με αν έχουν δυνατότητα ελεγχόμενης κίνησης ή όχι), μπορούν να χαρακτηριστούν \textit{Ενεργές (Active Force Closure)} ή \textit{Παθητικές (Passive Form Closure)} αντίστοιχα. Σε περίπτωση που υπάρχουν ενεργά και παθητικά στοιχεία (όπως για παράδειγμα ένας περιορισμός από κινητικά αδρανές στοιχείο του περιβάλλοντος) που δρουν ως προς διαφορετικές κατευθύνσεις, τότε η κλειστή λαβή ως προς δύναμη χαρακτηρίζεται \textit{υβριδική (Hybrid Force Closure)}~\cite{Yoshikawa2010}. Εξ' ορισμού η κλειστότητα ως 
προς τη μορφή προκύπτει μόνο υπό παθητικά περιοριστικά στοιχεία.\\

\begin{figure}[htbp] %  figure placement: here, top, bottom, or page
   \centering
   \includegraphics[width=0.8\textwidth]{images_kefalaio3/Closure_Types} 
   \caption{Τύποι κλειστότητας για κίνηση στο επίπεδο. (α) Παθητική Κλειστότητα ως προς τη μορφή.  (β) Παθητική Κλειστότητα ως προς τη Δύναμη. (γ) Ενεργητική Κλειστότητα. (δ) Υβριδική Ενεργητική/Παθητική Κλειστότητα~\cite{Yoshikawa2010}}
   \label{fig:Closure_Types}
\end{figure}


\textbf{Στατική Ισορροπία Λαβής}
\vspace{1ex}
 
 Μία λαβή μπορεί να ισορροπήσει όταν ο κυρτός φλοιός (Convex Hull) τον οποίον συνθέτουν τα διανύσματα γενικευμένων δυνάμεων που ασκούν τα δάκτυλα στο αντικείμενο περιλαμβάνει το σημείο μηδέν της διανυσματικής βάσης.\\
 
 \textbf{Λαβή Ακροδακτύλων (Fingertip Grasp)}
\vspace{1ex}
 
 Η λαβή να είναι σε θέση να αντισταθεί σε οποιαδήποτε τυχαία εξωτερική δύναμη έχοντας μοναδικά σημεία επαφής αυτά μεταξύ ακροδακτύλων αντικειμένου.\\
 
 
% \textbf{Λαβή Δύναμης (Power Grasp)}
%\vspace{1ex}
% 
%Αναμένουμε επίσης απόριψη εξωτερικών διαταραχών αλλά αυτή τη φορά οι επαφές μεταξύ ρομποτικού στοιχείου και αντικειμένου είναι πολλαπλές και δεν περιορίζονται στα ακροδάκτυλα.\\
 
 \textbf{Λαβή Ισχύος (power grasp ή enveloping grasp)}
\vspace{1ex}
 
Αναφέρεται στο είδος εκείνο της λαβής το οποίο χρησιμοποιείται για την σταθεροποίηση--πάκτωση αντικειμένων με χρήση πολλών σημείων--επιφανειών επαφής για μεγιστοποίηση της ικανότητας ως προς μεγάλα φορτία και τη σταθερή προσάρτησή τους. \\

%Άλλες κατηγορίες περιγραφής του μοντέλου της λαβής που συναντούμε στην έρευνα είναι αυτές της λαβής των ακροδακτύλων(fingertip grasp) και της περιβάλλουσας λαβής(enveloping grasp). Στη πρώτη αναμένουμε η λαβή να είναι σε θέση να αντισταθεί σε οποιαδήποτε τυχαία εξωτερική δύναμη έχοντας μοναδικά σημεία επαφής αυτά μεταξύ ακροδακτύλων αντικειμένου. Στη περίπτωση του enveloping grasp αναμένουμε επίσης απόρριψη εξωτερικών διαταραχών αλλά αυτή τη φορά οι επαφές μεταξύ ρομποτικού στοιχείου και αντικειμένου είναι πολλαπλές και δεν περιορίζονται στα ακροδάκτυλα.\\
 
%\begin{itemize}
%\item \textit{Passive Form Closure}\\Ο χαρακτηρισμός αυτός αφορά τη περίπτωση όπου παρ' ότι εφαρμόζονται μεταφορικές ή περιστροφικές δυνάμεις σε ένα 
%αντικείμενο υπό λαβή με σημειακές επαφές χωρίς τριβή, αυτό δεν μετακινείται μόνο εξαιτίας αποκλειστικά της δομής της λαβής ή του 
%μηχανισμού περιορισμού. Οι ασκούμενες δυνάμεις σε αυτή τη περίπτωση μπορεί να είναι οποιεσδήποτε.
% \item \textit{Passive Force Closure}\\Σε αυτή τη περίπτωση αντικείμενου υπό λαβή με σημειακές επαφές υπό τριβή όταν ασκούνται εσωτερικές δυνάμεις από τα 
%ακροδάκτυλα τότε οποιαδήποτε εξωτερική δύναμη μικρότερη από αυτή εξισορροπείται και δεν μετακινεί το αντικείμενο.
% \item \textit{Active Force Closure}\\Σε αυτή τη κατηγορία ανήκει και η περίπτωση αντικειμένου σε λαβή από δύο ρομποτικά δάκτυλα με δύο περιστροφικές
%αρθρώσεις. Με τις κατάλληλες ροπές στις αρθρώσεις, τυχαίες δυνάμεις επαφής στα ακροδάχτυλα και άρα τυχαία 
%συνιστάμενη δύναμη και ροπή μπορούν να ασκηθούν στο αντικείμενο το οποίο σημαίνει ότι τα δάκτυλα μπορούν να
%μετακινήσουν το αντικείμενο σε τυχαία κατεύθυνση.
%  \item \textit{Hybrid Force Closure}\\Όπου η κατεύθυνση του active force closure και του passive force closure συνυπάρχουν.
%  \end{itemize}
%  
 
%\textbf{Ποιότητα Λαβής}
%\vspace{1ex}
% 
%\textbf{Μήτρα Μετασχηματισμού Λαβής, Grasp Transformation Matrix}
%\vspace{1ex}
%


\textbf{Πρόβλημα Κατανομής Δυνάμεων (Force Distribution Problem)}
\vspace{1ex}

Ένα πολύ βασικό πρόβλημα στον επιδέξιο ρομποτικό χειρισμό είναι η επιλογή κατάλληλων δυνάμεων λαβής έτσι ώστε να αποφευχθεί, ή να ελαχιστοποιηθεί, ο κίνδυνος, ολίσθησης του αντικειμένου. Οι δυνάμεις λαβής (grasping) ή αλλιώς ονομαζόμενες εσωτερικές δυνάμεις, βρίσκονται στο μηδενικό χώρο του μητρώου λαβής (null space του grasp matrix). Οι δυνάμεις των επαφών οι οποίες δεν είναι άμεσα εσωτερικές,  επηρεάζουν την ισορροπία του αντικειμένου, και αναφέρονται ως δυνάμεις χειρισμού. Το πρόβλημα της επιλογής ροπών στις αρθρώσεις έτσι ώστε να δημιουργηθούν οι κατάλληλες δυνάμεις χειρισμού για την εργασία, και παράλληλα να υπάρχουν οι εσωτερικές δυνάμεις οι οποίες εγγυώνται την αποφυγή απώλειας στήριξης μέσω ικανοποίησης των συνθηκών του μοντέλου τριβής, αναφέρεται ως πρόβλημα κατανομής δυνάμεων (force distribution problem). Αυτό είναι κοινό πρόβλημα και σε άλλες περιοχές της ρομποτικής επιστήμης όπως η ρομποτική βάδιση, ο συνεργατικός ή υπό περιορισμούς χειρισμός. Μία σημαντική ιδιότητα πάνω στην οποία βασίζεται το πρόβλημα της μη γραμμικής βελτιστοποίησης των περιορισμών, και στο οποίο εδράζεται το πρόβλημα της κατανομής δυνάμεων, είναι η κυρτότητα. Δεδομένης της ικανοποίησης αυτής της ιδιότητας καθίσταται δυνατή η αποδοτική εύρεση λύσεων στο πολύπλοκο αυτό πρόβλημα. Για το ίδιο πρόβλημα έχουν προταθεί και αριθμητικές λύσεις επαναληπτικής μορφής μέσω της ολοκλήρωσης ομαλών διαφορικών εξισώσεων (ODE).  Σημαντικό για τη διατύπωση του προβλήματος της βελτιστοποίησης είναι η διαπίστωση ότι οι μη γραμμικοί περιορισμοί για τη τριβή μπορούν να διατυπωθούν και ως κατάλληλα θετικά ορισμένοι πίνακες. Αυτή η διατύπωση των περιορισμών σε μορφή πίνακα οδήγησε και σε επέκταση μετασχηματίζοντας το πρόβλημα σε τυπικό πρόβλημα ανισοτήτων πινάκων (LMI) όπου υπάρχουν έτοιμες ώριμες λύσεις υλοποιημένες και διαθέσιμες σε ευρέως διαθέσιμο λογισμικό.

\section{Περιγραφή Συστήματος}

\subsection{Μοντελοποίηση}
\label{subsec:Modeling}
Το σύστημα ρομποτικού χειρισμού αποτελεί ένα πολύπλοκο σύστημα μη γραμμικής δυναμικής. Επιπρόσθετα, τα υποσυστήματά, σύνδεσμοι - αρθρώσεις, μπορεί να είναι συζευγμένα. Αυτό οδηγεί αναπόφευκτα σε ιδιαίτερα εξελιγμένη μοντελοποίηση και κατ΄ επέκταση τεχνικές ελέγχου. Επίσης κατά την επαφή με το περιβάλλον στο σύστημα εισάγεται κινηματικός περιορισμός μέσω μίας διαδικασίας μηχανικών παραμορφώσεων. Αυτή η παραμόρφωση εξαρτάται από τη σκληρότητα -- ακαμψία του αντικειμένου καθώς και από τη σκληρότητα--ακαμψία και το σχήμα του τελικού στοιχείου δράσης κάθε ρομποτικής διάταξης. Δεδομένης αυτής της αλληλεπίδρασης, αναμένεται να δημιουργηθεί μία δύναμη αντίδρασης στο τελικό στοιχείο δράσης η οποία διοχετεύεται σε κάθε σύνδεσμο της ρομποτικής διάταξης.\\

\textbf{Υβριδική Μοντελοποίηση Δυναμικού Συστήματος Λαβής}
\vspace{1ex}

Ο επιδέξιος έλεγχος και χειρισμός αντικειμένων από  ρομποτική λαβή πολλών δακτύλων συνδυάζει χαρακτηριστικά από δύο τύπους αλληλεπιδρώντων δυναμικών συστημάτων. Από τη μία έχουμε τη πολύπλοκη δυναμική πολλών σωμάτων (multibody dynamics), η οποία μοντελοποιείται από ένα σύνολο από μη γραμμικές διαφορικές εξισώσεις, υποκείμενες σε ολονομικούς κινηματικούς και δυναμικούς περιορισμούς. Αυτό το κομμάτι επάγεται στη θεωρία των συνεχών μεταβαλλόμενων δυναμικών συστημάτων, \textit{Continuous Variable Dynamic Systems -- CVDS}. 

Από την άλλη έχουμε ένα σύνολο διακριτών φαινόμενων που περιγράφουν την κατάσταση των επαφών (\textit{discrete grasp states of fingers}) τα οποία περιγράφονται από τη θεωρία των δυναμικών συστημάτων διακριτών -- γεγονότων (discrete event dynamic systems - DEDS). 

Για τη μοντελοποίηση συστημάτων που συνδυάζουν διακριτή και συνεχή δυναμική μπορεί να γίνει χρήση της θεωρίας \textit{υβριδικών δυναμικών συστημάτων (Hybrid Dynamical Systems})~\cite{HybridDynamicalSystems_Savkin_Evans}, εισάγοντας την αντίστοιχη μοντελοποίηση--ανάλυση αλλά και τις σχετικές τεχνικές ελέγχου. 

Για τα ρομποτικά συστήματα δύναται να εισαχθεί ένας πρόσθετος φορμαλισμός στη περιγραφή αυτών με σκοπό να ειδικεύσει τις γενικές θεωρίες και περιγραφές της θεωρίας των υβριδικών δυναμικών συστημάτων για συστήματα \textit{μηχατρονικής πολλαπλών επαφών (Mechatronic Multicontact Systems})~\cite{SchleglBussSchmidt2002}. 

Κεντρικό στοιχείο της περιγραφής των υβριδικών δυναμικών συστημάτων αποτελεί το \textit{υβριδικό μοντέλο κατάστασης (Hybrid State Model--HSM}), στο οποίο η εξέλιξη του συστήματος στο χρόνο δίνεται κάθε φορά είτε από τη διαφορική εξίσωση συνεχούς χρόνου, στη περίπτωση που η κατάλληλη συνάρτηση επιλογής δίνει τιμή διάφορη του μηδενός, είτε από τη συνάρτηση διακριτών καταστάσεων για μηδενική τιμή της συναρτήσεως επιλογής. Αυτή η προσέγγιση περιγραφής -- ανάλυσης -- ελέγχου διά μέσου της θεωρίας των υβριδικών δυναμικών συστημάτων αποτελεί και το υψηλότερο επίπεδο φορμαλισμού που μπορεί κανείς να επιτύχει για την ολιστική περιγραφή του συστήματος της ρομποτικής λαβής~\cite{SchleglBussSchmidt2002}.

Η Υβριδική περιγραφή του δυναμικού συστήματος, αν και πλήρης, εισάγει ένα σημαντικό βαθμό πολυπλοκότητας και δυσκολίας στην ανάλυση και ανάπτυξη τεχνικών ελέγχου. Έτσι συνήθως οι περισσότερες εργασίες που έχουν να κάνουν με το κομμάτι του επιδέξιου χειρισμού εστιάζουν στο κομμάτι της συνεχούς δυναμικής που περιγράφει είτε το ρομποτικό σύστημα στον ελεύθερο χώρο είτε το σύστημα λαβής--αντικειμένου υπό σταθερή και κλειστή λαβή.

\begin{figure}[htbp] %  figure placement: here, top, bottom, or page
   \centering
   \includegraphics[width=0.8\textwidth]{images_kefalaio3/Hybrid_Modeling} 
   \caption{Hybrid System HDS~\cite{SchleglBussSchmidt2002}}
   \label{fig:Hybrid_System_Modeling}
\end{figure}

\subsection{Αισθητηριακή Σύνθεση}

Για τη πραγματοποίηση οποιασδήποτε ευφυούς δράσης, μέσω κάποιου σχήματος ελέγχου, είναι απαραίτητη η πληροφόρηση για τη κατάσταση του περιβάλλοντος. 

Για τα συστήματα ρομποτικού χειρισμού δύο είναι οι κυρίαρχες αισθητηριακές οδοί από το περιβάλλον και προκύπτουν από τα συστήματα όρασης και από τους αισθητήρες δύναμης/αφής.\\

\textbf{Συστήματα Όρασης}
\vspace{1ex}

Μία από τις πρώτες μεθόδους ελέγχου η οποία χρησιμοποιήθηκε στους ρομποτικούς χειριστές είναι η ανατροφοδότηση οπτικής πληροφορίας. Ο έλεγχος αυτού του τύπου έχει αποδειχτεί ένας αποτελεσματικός τρόπος για την ακριβή καθοδήγηση στον ελεύθερο χώρο, πάντα στο πλαίσιο του ρομποτικού χώρου εργασίας, χωρίς την εκ των προτέρων ακριβή μοντελοποίηση του συστήματος. Βασικές πρακτικές χρήσεις τέτοιων συστημάτων συναντώνται στις εργασίες σχεδιασμού τροχιάς και στον προσδιορισμό της γεωμετρίας άγνωστων αντικειμένων. Δύο διατάξεις είναι πιο συνήθεις, αυτή της τοποθέτησης του οπτικού στοιχείου σε σταθερό σημείο στο χώρο και αυτή της τοποθέτησης στο τελικό στοιχείο δράσης.

Πρακτικά η αποκόμιση επαρκούς πληροφορίας για το βάθος με μία μόνο μέτρηση είναι δύσκολη. Έτσι για την μέτρηση σε 3Δ χρησιμοποιούνται τεχνικές όπως, σύνθεση πολλαπλών λήψεων και η στερεοσκοπική όραση (stereovision).  Γενικά οι προσεγγίσεις για τον έλεγχο μέσω όρασης μπορούν να χωριστούν σε δύο κατηγορίες~\cite{KhalilPayeur2011}. 
\begin{itemize}
\item Τεχνικές βασιζόμενες στη θέση, όπου ένα σύνολο εικόνων αρχικοποιούνται μαζί με ένα γνωστό μοντέλο κάμερας για την εξαγωγή πληροφορίας ως προς την θέση/προσανατολισμό στο 3Δ χώρο. Οι μεταβλητές υπό έλεγχο είναι η καρτεσιανή θέση και προσανατολισμός του αντικειμένου. Στη περίπτωση όπου η κάμερα είναι σε σταθερό σημείο και η θέση/προσανατολισμός του αντικειμένου είναι υπό έλεγχο, οι μεταβλητές που περιγράφουν την θέση και το προσανατολισμό ανακατασκευάζονται από τις διαθέσιμες εικόνες. Συμπερασματικά, η ανίχνευση του αντικειμένου μπορεί να πραγματοποιηθεί υπολογίζοντας το σφάλμα στο 3Δ χώρο, και η θέση του αντικειμένου μπορεί να εξαχθεί χρησιμοποιώντας τις πληροφορίες της εικόνας και ένα βαθμονομημένο μοντέλο κάμερας.
\item Τεχνικές βασιζόμενες στην εικόνα. Στη προσέγγιση αυτή οι μεταβλητές υπό έλεγχο ορίζονται κατευθείαν ως χαρακτηριστικά στο χώρο της εικόνας και έτσι δεν είναι απαραίτητη η πλήρης 3Δ ανακατασκευή της σκηνής. Η ανίχνευση των αντικειμένων με τη συγκεκριμένη τεχνική πραγματοποιείται υπολογίζοντας το σφάλμα στο χώρο της εικόνας και εφαρμόζοντας έλεγχο ο οποίος εγγυάται ότι το σφάλμα αυτό θα μειώνεται ασυμπτωτικά στο μηδέν. Για μία σταθερή κάμερα η Ιακωβιανή μήτρα της εικόνας μπορεί να υπολογισθεί με τη χρήση μοντέλου της κάμερας. Εξαιτίας των παραμορφώσεων που εισάγονται στην εικόνα η ταυτοποίηση χαρακτηριστικών δεν είναι ακριβής. Ακόμα χειρότερα αποτελέσματα εισάγονται στη διάταξη κάμερας πάνω στο ρομποτικό τελικό στοιχείο δράσης.
\end{itemize}

Τελικά και οι δύο τεχνικές δεδομένης της ανακρίβειας τους στον προσδιορισμό της θέσης και του προσανατολισμού, κρίνονται γενικά μη ενδεδειγμένες για την χρησιμοποίηση τους στη επίτευξη και διατήρηση επαφής με την επιφάνεια του αντικειμένου~\cite{KhalilPayeur2011}. 

Γενικά όμως δε θα πρέπει να απορρίψουμε κατηγορηματικά  και τεχνικές οι οποίες στοχεύουν στην πραγματοποίηση ευσταθούς λαβής για άγνωστο αντικείμενο βασιζόμενες αποκλειστικά σε οπτική πληροφορία. Για παράδειγμα στην ~\cite{WangJiangLiCaiLiu2005} πραγματοποιείται ανακατασκευή του μοντέλου του άγνωστου αντικειμένου διά μέσου ειδικά προσαρμοσμένου laser 3D scanner στο καρπό του ρομποτικού βραχίονα με ταυτόχρονη υιοθέτηση τεχνικής βέλτιστης επιλογής λαβής βάσει κριτηρίων κλειστότητας καθώς και κίνησης εντός κώνου τριβής αυτοματοποιώντας τη διαδικασία αρπαγής αντικειμένου. Στη περίπτωση δύο εύκαμπτων δακτύλων ένα κριτήριο ποιότητας για την αρπαγή του αντικειμένου βάσει οπτικής πληροφορίας συνίσταται από ένα όρο ο οποίος ελαχιστοποιεί την απόσταση των άκρων των δακτύλων παράλληλα με ελαχιστοποίηση της απόστασης του γεωμετρικού κέντρου του αντικειμένου από τον άξονα που συνδέει τα ακροδάχτυλα και από ένα δεύτερο όρο που στοχεύει στην εφαρμογή δύναμης από τα δάκτυλα όσο το δυνατόν πιο κάθετα στην επιφάνεια αυτού~\cite{Yoshikawa2010}.\\

Το σύστημα όρασης γενικά πάντως φαίνεται να βρίσκει συμπληρωματική εφαρμογή στο πρώτο στάδιο, αυτό της προσέγγισης του αντικειμένου και ίσως και του πρώιμου σχηματισμού λαβής~\cite{KhalilPayeur2011}. Εφόσον το ρομποτικό στοιχείο δράσης φτάσει σε μία κατάλληλη απόσταση, η διαδικασία της επίτευξης και της βελτιστοποίησης ευσταθούς επαφής, πραγματοποιείται με χρήση πληροφοριών που παρέχονται σε πραγματικό χρόνο από τους αισθητήρες αφής και δύναμης. Οι στρατηγικές ελέγχου που υιοθετούνται και κάνουν χρήση των αισθητήρων δύναμης, αφής, στοχεύουν συνήθως στην ελαχιστοποίηση των εσωτερικών δυνάμεων(δυνάμεις λαβής) ή στην βελτιστοποίηση της θέσης/προσανατολισμού του αντικειμένου και τελικά στην επίτευξη επιδέξιου χειρισμού. Στόχος του όλου συστήματος αποτελεί η αυτονόμηση του συστήματος χειρισμού και ο έλεγχος αυτού μόνο στο υψηλό επίπεδο απ’ ευθείας ελέγχου του αντικειμένου.\\
  
\textbf{Αισθητήρες Γενικευμένων Δυνάμεων -- Αφής}
\vspace{1ex}

Οι αισθητήρες δύναμης οι οποίοι είναι εμπορικά διαθέσιμοι εγκαθίστανται συνήθως στον αντίστοιχο ρομποτικό καρπό, ή στους τένοντες του ρομποτικού χεριού. Συνήθως μετρούν τις δυνάμεις και τις ροπές οι οποίες αναπτύσσονται στο ρομποτικό χέρι κατά την αλληλεπίδραση με το περιβάλλον. Το μεγαλύτερο μέρος της διάταξης αισθητήρων τέτοιου τύπου αποτελείται από μετατροπείς οι οποίοι ανιχνεύουν κάποια γεωμετρική μεταβολή - παραμόρφωση κάποιου κατάλληλα σχεδιασμένου -- τοποθετημένου στοιχείου, ως συνάρτηση κάποιας ασκούμενης δύναμης--ροπής.\\

Οι αισθητήρες αφής τοποθετούνται συνήθως στην επιφάνεια η οποία προορίζεται για την απ’ ευθείας επαφή, κυρίως δηλαδή τα ακροδάχτυλα αλλά και άλλα εσωτερικά σημεία των δακτύλων και της παλάμης. Η μέτρηση αφορά την ασκούμενη πίεση η οποία παρατηρείται κατά την αλληλεπίδραση. Αυτή πραγματοποιείται μέσω μίας ηλεκτρονικής διάταξης η οποία περιλαμβάνει ένα επίπεδο συμμετρικό σύμπλεγμα μικρότερων αισθητηριακών στοιχείων ανίχνευσης της πίεσης τα οποία συνολικά μας δίνουν μία χαρτογράφηση των ασκούμενων πιέσεων. Οι πιο εξελιγμένοι από τους αισθητήρες αυτούς είναι σε θέση να μας δώσουν πλήρη εικόνα για το 6D διάνυσμα δυνάμεων της επαφής.\\
 
 Είναι ιδιαίτερα σημαντικό να αναφέρουμε σε αυτό το σημείο ότι οι παραδοσιακοί αισθητήρες δύναμης αποδίδουν θορυβώδη σήματα, δύσκολα στην επεξεργασία~\cite{KimYiOhSuh2003}. Αυτό το γεγονός αποτελεί και έναυσμα για τη πραγματοποίηση εναλλακτικών τεχνικών επιδέξιου χειρισμού που δε κάνουν αποκλειστική ή και καθόλου χρήση αισθητήρων δύναμης στα σημεία επαφής. 
 
 Η τεχνολογία αισθητήρων δύναμης--αφής φαίνεται πάντως να εξελίσσεται, με την ερευνητική προσπάθεια να δίνει καινούριες λύσεις, προσφέροντας τη δυνατότητα για ακριβή γνώση των δυναμικών χαρακτηριστικών της επαφής σε πραγματικό χρόνο με άμεσο αποτέλεσμα την εφαρμογή ελέγχου άμεσης δύναμης. Μία ενδιαφέρουσα προσέγγιση είναι η ανάπτυξη οπτικών αισθητήρων αφής υψηλής ακρίβειας ικανών για μέτρηση των κάθετων αλλά και ταυτόχρονα των εφαπτόμενων δυνάμεων οι οποίοι είναι σε θέση υπό κατάλληλο σχήμα ελέγχου να αποκτήσουν κατά τη πρώτη προσέγγιση του αντικειμένου πληροφορία για την ακαμψία αυτού προσαρμόζοντας τις παραμέτρους δυναμικού ελέγχου του χειρισμού στη συνέχεια, προσφέροντας εύρωστο επιδέξιο χειρισμό για μία μεγάλη γκάμα αντικειμένων διάφορων μηχανικών ιδιοτήτων~\cite{YussofOhka2009}. 
 
 Μία ακόμα πολλά υποσχόμενη τεχνολογία αισθητήρων αφής είναι σε θέση να αποκομίσει πληροφορία για την υφή των αντικειμένων, με τρόπο παρόμοιο με τον άνθρωπο, τρίβοντας τον αισθητήρα ο οποίος είναι προσαρμοσμένος στο κάθε ακροδάχτυλο πάνω στην επιφάνεια του αντικειμένου~\cite{10.3389/fnbot.2012.00004}. 
 
 Τέλος να σημειώσουμε ότι η τεχνική της κατάλληλης ενσωμάτωσης αισθητήρων δύναμης--αφής σε ρομποτικά χέρια με εύκαμπτες επιφάνειες αποτελεί μία ιδιαίτερη πρόκληση δεδομένου ότι η λήψη αισθητηριακής πληροφορίας δεν πρέπει να επηρεάζει άμεσα τη διαδικασία χειρισμού~\cite{LottiTiezziVassuraBiagiottiMelchiorri2004}.   

\section{Μεθοδολογίες Ελέγχου Ρομποτικής Λαβής}

\subsection{Έλεγχος Δύναμης}

Δύο βασικές κατηγορίες ελέγχου που προκύπτουν για μία ρομποτική κινηματική αλυσίδα είναι ο  “άμεσος έλεγχος δύναμης” και ο “έμμεσος έλεγχος δύναμης” ο οποίος επιτυγχάνει έλεγχο δύναμης μέσω κινηματικού ελέγχου. Οι βασικές πρακτικές υλοποιήσεις των δύο αυτών κατηγοριών επιτυγχάνονται με τον υβριδικό έλεγχο θέσης/δύναμης και με τον έλεγχο εμπέδησης αντίστοιχα.\\

\textbf{Υβριδικός Έλεγχος θέσης/δύναμης}
\vspace{1ex}

Ο υβριδικός έλεγχος θέσης/δύναμης, προσπαθεί να αποσυζεύξει τις κατευθύνσεις κατά τις οποίες πραγματοποιείται έλεγχος δύναμης και θέσης. Η κατεύθυνση κατά την οποία δεν υπάρχει περιορισμός αντιμετωπίζεται με έλεγχο θέσης και η κατεύθυνση κατά την οποία υπάρχει περιορισμός--επαφή με έλεγχο δύναμης. Έτσι στο τελικό σχέδιο ελέγχου υπάρχουν δύο παράλληλοι βρόχοι ελέγχου. Πρακτικά η εναλλαγή μεταξύ αυτών των δύο βρόχων μπορεί να μη γίνεται αρκετά γρήγορα για την αντιμετώπιση των αλλαγών του περιβάλλοντος~\cite{Yoshikawa2010}.\\

\textbf{Έλεγχος Εμπέδησης -- Impedance Control}
\vspace{1ex}

Σε αντίθεση με τον υβριδικό έλεγχο θέσης/δύναμης, ο έλεγχος εμπέδησης συνδυάζει τον έλεγχο θέσης/δύναμης. Αυτή η προσέγγιση στοχεύει στην ομαλοποίηση της ακαμψίας του ρομποτικού χειριστή με τον ορισμό της επιθυμητής εμπέδησης στο τελικό στοιχείο δράσης. Με διαφορετική ματιά η μέθοδος αυτή στοχεύει στον έλεγχο της θέσης και της δύναμης στον ίδιο χρόνο εκφράζοντας την επιθυμητή εργασία ως την επίτευξη κατάλληλα ορισμένης επιθυμητής εμπέδησης. Η τελική σύνθετη μηχανική αγωγιμότητα (admittance) του περιβάλλοντος, καθώς και η τελική θέση και ασκούμενη δύναμη θα είναι συνάρτηση της ρομποτικής εμπέδησης. Ο έλεγχος εμπέδησης θεωρείται η πιο κατάλληλη λύση για την αντιμετώπισης αλληλεπιδράσεων σε μη δομημένα περιβάλλοντα~\cite{KhalilPayeur2011}. Προβλήματα εντοπίζονται λόγω ύπαρξης λαθών στη μοντελοποίηση ή εξαιτίας της μη μοντελοποιημένης δυναμικής ο ελεγκτής προκαλεί αναίτια δράση. 

Το σχήμα ελέγχου της εμπέδησης έχει δύο εκφράσεις. 

\begin{itemize}
\item Έλεγχος Σύνθετης Μηχανικής (Impedance Control). Η ρομποτική διάταξη αντιδρά στην απόκλιση από την δεδομένη τροχιά παράγοντας δυνάμεις.
\item Έλεγχος Σύνθετης Μηχανικής Αγωγιμότητας (Admittance Control). Η ρομποτική διάταξη αντιδρά σε εξωτερικές δυνάμεις με απόκλιση από την επιθυμητή τροχιά διατηρώντας τις δυνάμεις αλληλεπίδρασης σε επιθυμητές τιμές.
\end{itemize}

Ειδικές περιπτώσεις Impedance και Admittance control αποτελούν οι τεχνικές ελέγχου μηχανικής ακαμψίας (stiffness control) και 
ελέγχου μηχανικής συμμόρφωσης (ή ευκαμψίας, compliance control), αντίστοιχα, όπου μας ενδιαφέρει μόνο η στατική σχέση μεταξύ της θέσης -- προσανατολισμού
του στοιχείου δράσης και επιθυμητής κίνησης και η δύναμη -- ροπή της επαφής οι οποίες και 
λαμβάνονται υπόψη Αν η σχέση μεταξύ της δύναμης--ροπής της επαφής και των γραμμικών και 
γωνιακών ταχυτήτων του στοιχείου δράσης είναι τα μεγέθη που μας ενδιαφέρουν το σχήμα ελέγχου 
ονομάζεται έλεγχος απόσβεσης (damping control).  
\\

\textbf{Υβριδικός Έλεγχος Εμπέδησης}
\vspace{1ex}

Ο υβριδικός έλεγχος εμπέδησης συνδυάζει τον υβριδικό έλεγχο και τον έλεγχο εμπέδησης, με έναν εσωτερικό βρόχο ελέγχου αντίστροφης δυναμικής και ένα εξωτερικό βρόχο με στόχο την επίτευξη κατάλληλων επιθυμητών χαρακτηριστικών, όπως παρακολούθηση σημείου-στόχου, απόρριψη διαταραχών καθώς και θέματα ευρωστίας. Ανάλογα με το τι χρειάζεται να ελεγχθεί κάθε φορά το σχήμα αυτό μετονομάζεται πιο εξειδικευμένα, υβριδικός έλεγχος εμπέδησης/θέσης ή υβριδικός έλεγχος εμπέδησης/δύναμης. Για παράδειγμα, έλεγχος εμπέδησης ως προς τη δύναμη μπορεί να χρησιμοποιηθεί για την παραγωγή των εσωτερικών δυνάμεων εκείνων οι οποίες θα εγγυηθούν ότι δε θα χαθεί η επαφή με το αντικείμενο, ενώ ο έλεγχος εμπέδησης ως προς θέση μπορεί να τοποθετήσει τα δάκτυλα και άρα αντικείμενα στο επιθυμητό σημείο. 

Στο επίπεδο του ελεγκτή αυτού καθεαυτού, έχουν προταθεί διάφοροι κλασικοί καθώς και μοντέρνοι ελεγκτές από τη βιβλιογραφία της ρομποτικής επιστήμης για τον έλεγχο της κίνησης του χειριστή. Αυτοί ποικίλουν από τον κλασικό PID μέχρι και μη γραμμικές σύγχρονες προσεγγίσεις όπως αυτές των, “μεταβλητής δομής”, προσαρμοστικών και εύρωστων. Τελευταία η τάση στη ρομποτική κοινότητα μετατοπίζεται σε θέματα Τεχνητής Νοημοσύνης, όπως τα έμπειρα συστήματα, ασαφής λογική, νευρωνικά δίκτυα. Η βιομηχανία πάντως δείχνει να παραμένει πιστή σε αναλυτικές λύσεις, αν και απλοί αλγόριθμοι μάθησης αποτελούν πάντα μία πιθανή λύση για βιομηχανικές εφαρμογές, ιδιαίτερα στο επίπεδο της εργασίας.

\subsection{Έλεγχος Ακαμψίας σε επίπεδο Αντικειμένου -- Object Stiffness Control}
Ένα επιδέξιο χέρι έχει τη δυνατότητα να κρατάει οποιοδήποτε τυχαίο αντικείμενο καθώς και να αντιδρά σε τυχαίες κινήσεις και δυνάμεις οι οποίες μπορεί να δρουν πάνω στο αντικείμενο. Αυτές οι κινήσεις και δυνάμεις όντας συζευγμένες και βάσει κατάλληλου σχήματος ελέγχου μπορούν να καθορίσουν τη γενικευμένη ακαμψία (stiffness) ενός χειριζόμενου αντικειμένου καθώς και το αντίστοιχο κέντρο συμμόρφωσης αυτού.

Ενδιαφέρον παρουσιάζει η εφαρμογή αυτού του σχήματος ελέγχου της ακαμψίας του αντικειμένου έτσι ώστε η δύναμη αντίδρασης κατά την αλληλεπίδραση να καθοδηγήσει κάποια διαδικασία συναρμολόγησης υπομέρος κάποιας ευρύτερης εργασίας του ρομποτικού χειριστή. Πρακτικά ένα ρομποτικό χέρι μπορεί να χρησιμοποιηθεί στη διαδικασία παραγωγής σε σημεία συναρμολόγησης όπου μπορούν να παρέχουν δυνατότητες ενεργούς κίνησης μηχανικής συμμόρφωσης, η οποία παραδοσιακά γίνεται από έναν εγγενώς παθητικό χειριστή ή από ένα παθητικό στοιχείο κεντρικής συμμόρφωσης. Ένα ρομποτικό χέρι εξαιτίας των ιδιαίτερα χαμηλών αδρανειακών χαρακτηριστικών του μπορεί να προσφέρει δυνητικά υψηλότερο εύρος ζώνης συγκριτικά με ένα τυπικό χειριστή σειριακών συνδέσμων~\cite{GregoryStarr1992}.

Για την επιτυχή διαδικασία συναρμολόγησης, η συνιστάμενη δύναμη πάνω στο αντικείμενο πρέπει να είναι σωστά συζευγμένη με την κίνηση του αντικειμένου. Αυτό μπορεί να γίνει με διάφορους τρόπους, ένας εκ των οποίων είναι η εισαγωγή κατάλλη- λου σχήματος ελέγχου για το καθορισμό της μηχανικής εμπέδησης των αντικειμένων. Αυτή η τεχνική δεν έχει στόχο την επιδεξιότητα όπως την ορίσαμε υπό την έννοια της ικανότητας προσανατολισμού του αντικειμένου στο χώρο.

Αρκετές προσεγγίσεις ορίζουν έλεγχο μόνο ως προς το στατικό όρο του μοντέλου της εμπέδησης του αντικειμένου, αυτό της ακαμψίας(object stiffness control). Ο έλεγχος κινείται γύρω από το σημείο στατικής ισορροπίας παράγοντας δύναμη στο αντικείμενο μέσω των ακροδάχτυλων αναλογικά ως προς τη μετατόπιση από το σημείο ισορροπίας, ταυτόχρονα με μία λογική ελέγχου των εσωτερικών δυνάμεων (Grasp Force Optimaziation) προς αποφυγή απώλειας στήριξης. 

Μία τεχνική ελέγχου των εσωτερικών δυνάμεων που δρα στο πλαίσιο του ελέγχου της ακαμψίας της θέσης του αντικειμένου, θεωρεί τις δυνάμεις αυτές παράλληλες ως προς τους αντίστοιχους άξονες που ενώνουν τα σημεία επαφής των δακτύλων και τις ενσωματώνει στο γενικευμένο διάνυσμα δυνάμεων του αντικειμένου. Στη συνέχεια η μήτρα λαβής επεκτείνεται ως προς τις γραμμές για την απεικόνιση από το χώρο των δυνάμεων των ακροδάχτυλων στο χώρο των εσωτερικών δυνάμεων στο αντικείμενο. Ο έλεγχος ακαμψίας του αντικειμένου μέσω προσδιορισμού κατάλληλων ροπών στις αρθρώσεις γίνεται αντιστρέφοντας την τετραγωνική πλέον μήτρα λαβής και στη συνέχεια με τη χρήση της Ιακωβιανής μήτρας. Αυτή η προσέγγιση βασιζόμενη σε μία πειραματική διάταξη μετάδοσης κίνησης μέσω τενόντων έχει ήδη δώσει ικανοποιητικά αποτελέσματα ~\cite{GregoryStarr1992}. 

\subsection{Έλεγχος Ακαμψίας σε επίπεδο Αντικειμένου με Πλεονασματικές Κινηματικές Διατάξεις}

Στη περίπτωση του ελέγχου της ακαμψίας ως προς το αντικείμενο με τη χρήση δακτύλων πλεοναζόντων βαθμών ελευθερίας απαιτείται η διερεύνηση κατάλληλων τεχνικών για την ικανοποίηση επιθυμητών επιμέρους κριτηρίων. Μία ενδιαφέρουσα τεχνική ελέγχου της ακαμψίας του αντικειμένου αποδομεί το πρόβλημα, προσδιορίζοντας αρχικά, ένα στατικό μοντέλο με το οποίο συνδέεται η ακαμψία του χώρου του αντικειμένου με τις επιμέρους παραμέτρους ακαμψίας των ακροδάχτυλων. Δεδομένης της επιθυμητής ακαμψίας του αντικειμένου, το πρόβλημα ανάγεται στον προσδιορισμό των παραμέτρων ακαμψίας των ακροδάχτυλων. Προς αυτή τη κατεύθυνση το αρχικό μοντέλο μπορεί να έρθει σε γραμμική μορφή υποθέτοντας ότι η μήτρα ακαμψίας έχει αποκλειστικά διαγώνια στοιχεία μη μηδενικά και σχηματίζοντας με αυτά ένα διάνυσμα συντελεστών ακαμψίας. 

Ο υπολογισμός των στοιχείων ακαμψίας των ακροδάχτυλων γίνεται βάσει αλγόριθμου βελτιστοποίησης όπου απαιτείται ελαχιστοποίηση της νόρμας της διαφοράς των στοιχείων της ακαμψίας με την ελάχιστη τιμή που αυτά μπορούν να έχουν ικανοποιώντας, ως ισοτικό περιορισμό, τη γραμμική εξίσωση του συστήματος. Ο αλγόριθμος αυτός ονομάζεται, αλγόριθμος σύνθεσης ακαμψίας ακροδάχτυλων (Fingertip stiffness synthesis algorithm, FSS) και αποτελείται από δύο στάδια.
\\

\textbf{Αλγόριθμος FSS}
\begin{itemize}
\item Eξασφαλίζει αρχικά την καταλληλότητα της επιλογής των παραμέτρων ακαμψίας του αντικειμένου έτσι ώστε να προκύψει ευστάθεια για το σύστημα. 
\item  Yπολογίζει τα στοιχεία της μήτρας ακαμψίας βάσει της κλασικής λύσης για προβλήματα ελαχίστων τετραγώνων με ισοτικό περιορισμό. 
\end{itemize}

Στη συνέχεια ο πρόβλημα ανάγεται στον προσδιορισμό της ακαμψίας των αρθρώσεων βάσει της υπολογισθείσας μήτρας ακαμψίας στο χώρο των ακροδάχτυλων. Για το σκοπό αυτό, έχει αναπτυχθεί τεχνική ελέγχου αποκαλούμενη: Ορθογώνιος Έλεγχος Αποδόμησης της Ακαμψίας (Orthogonal Stiffness Decomposition Control -- OSDC).\\

 \textbf{Αλγόριθμος OSDC}
 \\
 
Βάσει της στατικής εξίσωσης που συνδέει τις ροπές στις αρθρώσεις με την ασκούμενη δύναμη από τα ακροδάχτυλα, μέσω της Ιακωβιανής κινηματικής μήτρας, προκύπτει η σχέση μεταξύ της ακαμψίας των αρθρώσεων και της ακαμψίας στο άκρο του δακτύλου. Δεδομένου ότι τα δάκτυλα έχουν πλεονασματικούς βαθμούς ελευθερίας ακολουθείται τεχνική ορισμού του μηδενικού χώρου της απεικόνισης μέσω κατάλληλου μετασχηματισμού ομοιότητας. 
 
 Τελικά ο έλεγχος στο επίπεδο των αρθρώσεων περιλαμβάνει τον όρο της ακαμψίας με κέρδος την τιμή που υπολογίζεται βάσει του OSDC, ένα δυναμικό όρο απόσβεσης ο οποίος έχει κέρδος προσδιοριζόμενο από την αντίστροφη μήτρα του όρου της μήτρας ακαμψίας που απεικονίζει στο μη μηδενικό χώρο και ένα δυναμικό όρο βαρυτικής αντιστάθμισης.   
 
 Ολόκληρο το σχήμα ελέγχου το οποίο συνδυάζει τις επιμέρους τεχνικές FSS και OSDC Αποκεντρωμένος Έλεγχος Ακαμψίας σε επίπεδο Αντικειμένου (Decentralized..., DOSC) και είναι αποτελεσματικό ως προς τον αξιόπιστο καθορισμό της ακαμψίας στο επίπεδο του αντικειμένου με ρομποτικά δάκτυλα πλεοναζόντων βαθμών ελευθερίας~\cite{ChoiChungYoum1994}.
 
 Μία άλλη τεχνική προσδιορισμού της ακαμψίας των αρθρώσεων για δεδομένη διαγώνια μήτρα ακαμψίας των ακροδάχτυλων σε πλεονασματικές κινηματικές διατάξεις στηρίζεται στη τεχνική των επαυξημένων χώρων~\cite{KumazakiSvininLuoOdashimaHosoe2002}. Ουσιαστικά ορίζεται μία υποεργασία στο μηδενικό χώρο δράσης της Ιακωβιανής μήτρας της κινηματικής διάταξης αποσυζευγμένη από τη κύρια εργασία. Με αυτό το τρόπο μπορεί να απεικονιστεί ξεχωριστά η ακαμψία της κύριας εργασίας, που είναι η ακαμψία στο χώρο των ακροδάχτυλων, από την ακαμψία της υποεργασίας δίνοντας με αυτό το τρόπο μήτρα ακαμψίας των αρθρώσεων πλήρους τάξης.

\subsection{Έλεγχος Σύνθετης Μηχανικής Αγωγιμότητας -- Admittance Control}

Όπως ήδη έχουμε αναφέρει ο ρομποτικός έλεγχος σύνθετης μηχανικής αγωγιμότητας (admittance control) έχει ως στόχο τη παραγωγή κατάλληλων μετατοπίσεων της ρομποτικής διάταξης αντιδρώντας έτσι σε εξωτερικώς ασκούμενες δυνάμεις διατηρώντας τις δυνάμεις αλληλεπίδρασης σε επιθυμητές τιμές. Για το σύστημα της ρομποτικής λαβής αυτή η συμπεριφορά δυναμικής συμμόρφωσης είναι επιθυμητό να υιοθετηθεί σε σχέση με το αντικείμενο υπό χειρισμό.  

Στη περίπτωση που ο έλεγχος επίτρεψης περιοριστεί μόνο στο προσδιορισμό των στατικών παραμέτρων του δυναμικού μοντέλου του κλειστού συστήματος τότε το σχήμα έλεγχου ονομάζεται έλεγχος συμμόρφωσης (compliance control). Αυτού του τύπου η στατική ανάλυση είναι γενικά χρήσιμη για την γενική εξαγωγή συμπερασμάτων για το σύστημα της λαβής ιδιαίτερα στη γραμμική προσέγγιση γύρω από ένα σημείο ισορροπίας ή κατά τις μικρές κινήσεις όπου οι αδρανειακοί όροι είναι μικροί.

Μία μεθοδολογία για την επίτευξη ελέγχου συμμόρφωσης στο επίπεδο του αντικειμένου συνίσταται στο κατάλληλο προσδιορισμό των μητρών ακαμψίας αρχικά ως προς τα ακροδάχτυλα -- σημεία επαφής και στη συνέχεια ως προς τις αρθρώσεις των δακτύλων όπως δηλαδή και στον έλεγχο ακαμψίας του αντικειμένου (object stiffness control) όπως περιγράφηκε στις προηγούμενες ενότητες. Στην~\cite{KimYiOhSuh2003} προτείνεται ένα σχήμα ελέγχου για την επίτευξη συμμόρφωσης στο επίπεδο του αντικειμένου με μεθοδολογία προσδιορισμού των μητρών ακαμψίας έτσι ώστε να επιτυγχάνεται ταυτόχρονα αποσύζευξη μεταξύ των δακτύλων αλλά και μεταξύ των αρθρώσεων κάθε δακτύλου. Αυτή η αποσύζευξη διευκολύνει τον έλεγχο του χεριού. Πιο συγκεκριμένα η υλοποίηση περιλαμβάνει δύο βασικά στάδια υλοποιούμενα με δύο διαφορετικούς αλγόριθμους αντίστοιχα.   

\begin{itemize}
\item
\textbf{Resolved Interfinger Decoupling Solver (RIFDS)}\\
Αρχικά ελέγχεται κατά πόσο ο επιθυμητός πίνακας ακαμψίας του αντικειμένου είναι δυνατό να πραγματοποιηθεί από την εκάστοτε ρομποτική διάταξη χειρισμού βάσει ευριστικού μοντέλου υλοποιούμενου με προκαθορισμένους πίνακες οι οποίοι λαμβάνουν υπόψη το μοντέλο των επαφών καθώς και τη κινηματική διάταξη βάσει των διαθέσιμων βαθμών ελευθερίας. Στη συνέχεια αφού υπολογιστεί ο πίνακας λαβής (Grasping Matrix) για το τρέχον χρονικό βήμα οι εξισώσεις που συνδέουν τις μήτρες ακαμψίας του αντικειμένου και των ακροδάχτυλων, οι οποίες προκύπτουν από τη στατική εξίσωση δυνάμεων -- ταχυτήτων της λαβής, μετασχηματίζονται κατάλληλα σε γραμμική σχέση με τον υπολογισμό πλέον της μήτρας ακαμψίας των ακροδάχτυλων να έχει μετατραπεί πλέον σε πρόβλημα γραμμικού προγραμματισμού. 
\item
\textbf{Resolved Interjoint Decoupling Solver (RIJDS)}\\
Σε αυτό το στάδιο στόχο αποτελεί ο προσδιορισμός των παραμέτρων ακαμψίας για κάθε άρθρωση των δακτύλων δεδομένου ότι ήδη έχει προσδιορισθεί η μήτρα ακαμψίας ως προς το ακροδάχτυλο με τον αλγόριθμο RIFDS. Με παρόμοιο τρόπο η μαθηματική σχέση που συνδέει τις δύο μήτρες ακαμψίας βάσει της Ιακωβιανής διαφορικής μήτρας μετασχηματίζεται σε γραμμική μορφή εξασφαλίζοντας αποσυζευγμένη συμπεριφορά μεταξύ των αρθρώσεων. Η τελική λύση προκύπτει με απλή αντιστροφή της μήτρας γραμμικής απεικόνισης αυτής της σχέσης.
\end{itemize}


\subsection{Υβριδικός Δυναμικός Έλεγχος Συστήματος Ρομποτικής Λαβής}

Όπως αναφέραμε σχετικά στο κεφάλαιο που πραγματεύεται το κομμάτι της μοντελοποίησης του συστήματος \ref{subsec:Modeling}, το υψηλότερο επίπεδο φορμαλισμού που μπορούμε να επιτύχουμε για την πλήρη περιγραφή του συστήματος της ρομποτικής λαβής -- αντικειμένου είναι αυτό των υβριδικών δυναμικών συστημάτων τα οποία συνδυάζουν δυναμική διακριτής και συνεχούς φύσεως. Οι Schleg, Buss \& Schmidt στην εργασία τους~\cite{SchleglBussSchmidt2002} επιχειρούν στο πλαίσιο αυτής της ολιστικής θεώρησης του προβλήματος να εισάγουν ένα σύνθετο σχήμα ελέγχου με το οποίο μπορούν να προσδώσουν στη ρομποτική λαβή σχεδόν το σύνολο των χαρακτηριστικών που μπορούν να τη χαρακτηρίσουν επιδέξια. Πιο αναλυτικά ο σχεδιασμός τους επιτρέπει στη ρομποτική λαβή την εύρωστη αρπαγή αντικειμένων, ελεύθερη τοποθέτηση σε θέση και προσανατολισμό, αλλαγή διάταξης λαβής με regrasping, αντιμετώπιση διαταραχών και ανακριβειών στη μοντελοποίηση.\\

Ο σύνθετος αυτός έλεγχος απαρτίζεται από τρία βασικά μέρη,

\begin{itemize}
\item
Γεννήτρια Υβριδικών τροχιών Αναφοράς -- Hybrid Reference Generator (HRG). Οι επιθυμητές τροχιές παράγονται με τη χρήση υβριδικών αυτόματων αναφοράς. Η υβριδική τροχιά συνδυάζει την επιθυμητή κατάσταση επαφής η οποία χαρακτηρίζεται από τον αριθμό των δακτύλων υπό επαφή, τον τύπο της επαφής, την επιθυμητή θέση των ακροδάχτυλων καθώς και τις αντίστοιχες ταχύτητες αυτών.  

\item
Υβριδικός Ελεγκτής -- Hybrid Controller (HC). Ο εσωτερικός αυτός βρόχος ελέγχου οδηγεί το σύστημα ως προς την επιθυμητή τροχιά αναφοράς. Στο διακριτό του τμήμα παράγει κατάλληλο διάνυσμα διακόπτη βάσει του διανύσματος διακριτής κατάστασης της λαβής, που στόχο έχει την εναλλαγή των impedance controllers. Στο συνεχές του κομμάτι αναλαμβάνει τη παραγωγή του σήματος λάθους που στη συνέχεια τροφοδοτεί τον επιλεγμένο ελεγκτή εμπέδησης.  
\item
Βελτιστοποίηση Δυνάμεων Λαβής \& Έλεγχος Εμπέδησης \\ Grasp Force Optimization (GFO) \& Impedance Control. Η τεχνική GFO που χρησιμοποιείται αναλύεται στην επόμενη ενότητα~\cite{BussScleg1997}. Ο ελεγκτής εμπέδησης διαθέτει μία σειρά από επιμέρους ελεγκτές εμπέδησης από τους οποίους επιλέγεται κάθε στιγμή ο κατάλληλος από τον υβριδικό ελεγκτή βάσει του διανύσματος που περιγράφει τη κατάσταση της λαβής κάθε χρονική στιγμή (βλ. σχήμα \ref{fig:Bank_Impedance})  εξασφαλίζοντας με αυτό το τρόπο την επιθυμητή δυναμική απόκριση του συστήματος για κάθε δεδομένη στιγμή. 
\end{itemize}

Η γνώση για την ασκούμενη δύναμη στο αντικείμενο, απαραίτητη για το στάδιο ελέγχου GFO, παρέχεται βάσει μετρήσεων από αισθητήρες γενικευμένων δυνάμεων στα ακροδάκτυλα 6D και με απεικόνιση αυτών στο χώρο των ασκούμενων δυνάμεων στο αντικείμενο μέσω της μήτρας μετασχηματισμού της λαβής. 

\begin{figure}[htbp] %  figure placement: here, top, bottom, or page
   \centering
   \includegraphics[width=0.8\textwidth]{images_kefalaio3/Impedance_Hybrid} 
   \caption{Bank of Impedance Controllers~\cite{SchleglBussSchmidt2002}}
   \label{fig:Bank_Impedance}
\end{figure}

%\subsection{Regrasping \& Finger Gaiting}
%\subsection{Sliding \& Rolling}

\subsection{Βελτιστοποίηση Δυνάμων Λαβής (Grasping Force Optimization -- GFO)}

Η άσκηση δυνάμεων λαβής από τη ρομποτική διάταξη χειρισμού στο αντικείμενο με βέλτιστο τρόπο αποτελεί μία από τις βασικές προϋποθέσεις για την επίτευξη εύρωστου επιδέξιου χειρισμού. Οι κλασικές προσεγγίσεις του συγκεκριμένου προβλήματος λειτουργούν υπό τις παραδοχές της σημειακής επαφής με τριβή και την αλληλεπίδραση συμπαγών αντικειμένων. Υπό αυτό το απλουστευτικό για την ανάλυση πρίσμα τα δύο βασικότερα σημεία που πρέπει να ικανοποιεί ο βέλτιστος έλεγχος των δυνάμεων λαβής είναι η κίνηση πάντα εντός του κώνου τριβής εξασφαλίζοντας την ικανή στήριξη του αντικειμένου και παράλληλα η ελαχιστοποίηση των εσωτερικών δυνάμεων υπό τη λογική της ελάχιστης προσπάθειας αλλά και της προστασίας του συνολικού συστήματος από φθορές. Είναι άμεσα προφανές ότι αυτοί οι στόχοι είναι αντιδιαμετρικοί και απαιτούν ένα συμβιβασμό. 

Ως προς το φορμαλισμό της περιγραφής του προβλήματος βελτιστοποίησης της άσκησης δυνάμεων λαβής, μεγάλη συνεισφορά έχει η διατύπωση των όρων του μοντέλου τριβής Coulomb για κάθε επαφή με τη μορφή θετικά ορισμένου πίνακα. Βάσει αυτού μπορεί να σχηματιστεί αντίστοιχη συνάρτηση κόστους η οποία περιγράφει τα κριτήρια βελτιστοποίησης που αναφέραμε παραπάνω. Μία τέτοια συνάρτηση κόστους θα μπορούσε να είναι για παράδειγμα το άθροισμα των στοιχείων του σταθμισμένου πίνακα όρων τριβής συν το άθροισμα του σταθμισμένου αντίστροφου πίνακα των όρων τριβής. Έτσι με επιλογή κατάλληλων πινάκων στάθμισης και ελαχιστοποίηση της συνάρτησης κόστους είναι δυνατό να ικανοποιήσουμε τα κριτήρια που περιγράψαμε. Η στάθμιση αυτή μάλιστα μπορεί να αλλάζει δυναμικά για την εφαρμογή σε τεχνική regrasping όπου απαιτείται ομαλή μετάβαση από το στάδιο της επαφής στο στάδιο της μη επαφής για κάθε δάκτυλο και αντίστροφα σε συνδυασμό με ένα σύνολο επιπρόσθετων γραμμικών περιορισμών ως προς τη μήτρα των όρων τριβής για βελτίωση της αποτελεσματικότητας της τεχνικής~\cite{BussScleg1997}. 

Ως προς το υπολογιστικό σκέλος αυτής της μεθόδου υπάρχουν πολλές προσεγγίσεις βάσει της κλασικής θεωρίας τεχνικών βελτιστοποίησης οι οποίες όμως δεν κρίνονται κατάλληλες για εφαρμογή σε πραγματικό χρόνο~\cite{BussScleg1997}. Στη περίπτωση που απαιτείται αντίδραση σε πραγματικό χρόνο είναι προτιμότεροι αλγόριθμοι επαναληπτικών τεχνικών αριθμητικών λύσεων οι οποίοι μάλιστα δεν αποκλείεται να να μειώνουν τις απαιτήσεις ως προς την ακρίβεια των λύσεων επ'ωφελεία της ταχύτητας του συστήματος, για την εξασφάλιση δηλαδή άμεσης απόκρισης.. Οι Buss και Schleg~\cite{BussScleg1997} προτείνουν ένα αλγόριθμο ο οποίος, αρχικά, δεδομένης της εξωτερικής δύναμης ως προς το αντικείμενο, αυξάνει με γραμμικό τρόπο τις εσωτερικές δυνάμεις, έτσι ώστε η συνολική δύναμη από κάθε δάκτυλο να βρίσκεται εντός κώνου τριβής και στη συνέχεια η ρύθμιση των εσωτερικών δυνάμεων γίνεται βελτιστοποιώντας τη συνάρτηση κόστους που περιγράψαμε παραπάνω με επαναληπτική αριθμητική μέθοδο.


\subsection{Προσχηματισμός Ρομποτικής Λαβής -- Preshaping Robot Hand}

Απαραίτητο για το χειρισμό ενός αντικειμένου αποτελεί το στάδιο κατά το οποίο η ρομποτική διάταξη προσεγγίζει το αντικείμενο σχηματίζοντας κατάλληλο σχήμα για τη σύναψη κλειστής λαβής. Η προσέγγιση αυτή, αν και διαισθητικά απλή, προϋποθέτει ένα συντονισμό σύνθετων υποεργασιών. 

\begin{itemize}
\item
Την αναγνώριση του αντικειμένου, της θέσης, του σχήματος και του μεγέθους του.
\item
Παραγωγή κατάλληλων σημάτων ελέγχου της κινηματικής διάταξης για τη παρακολούθηση της κίνησης του αντικειμένου.
\item
Το σχεδιασμό κατάλληλης λαβής βάσει του σχήματος και της μορφολογίας της επιφάνειας του αντικειμένου, την εφαρμογή αυτής και τελικά το χειρισμό του αντικειμένου.
\end{itemize}

Μία ενδιαφέρουσα προσέγγιση~\cite{LuoTakahimSugimotoSugimotoOdashimaHosoe2002} θέτει κατάλληλο σχήμα εικονικής εμπέδησης (virtual impedance) με δύο συνιστώσες. Η μία συνιστώσα ορίζεται ως η εικονική ακαμψία (virtual stiffness) για το κάθε άκρο των δακτύλων ως προς τη θέση του αντικειμένου κάθε χρονική στιγμή, εξασφαλίζοντας με αυτό τον τρόπο παρακολούθηση της κίνησης του αντικειμένου. Η δεύτερη συνιστώσα εικονικής ακαμψίας εξασφαλίζει ότι τα ακροδάχτυλα θα καλύψουν συμμετρικά το χώρο ``αγκαλιάζοντας" συμμετρικά το αντικείμενο με χρήση κατάλληλων συναρτήσεων, οι οποίες απωθούν όλα τα δάκτυλα μεταξύ τους, εξασφαλίζοντας με αυτό το τρόπο κλειστότητα. Αυτές οι δύο εικονικές εμπεδίσεις υπερτίθενται στο τελικό σήμα ελέγχου και απεικονίζονται στο χώρο των αρθρώσεων κατά το κλασικό στατικό μοντέλο με την ανάστροφη ιακωβιανή διαφορική μήτρα.\\

\newpage
\subsection{Άλλες Τεχνικές}

\textbf{Τεχνικές Μηχανικής Μάθησης}
\vspace{1ex}

Οι τεχνικές μηχανικής μάθησης κερδίζουν συνεχώς έδαφος στα ρομποτικά συστήματα είτε συμπληρωματικά με τις κλασσικές τεχνικές ελέγχου είτε και ως αυτοτελή συστήματα ευφυούς ελέγχου. 

Στην ~\cite{NguyenArimoto2002} παρουσιάζεται μία τεχνική ελέγχου για ένα σύστημα λαβής 2 δακτύλων 2 DOFs με εύκαμπτα ακροδάκτυλα με ικανότητα κύλισης ως προς την επιφάνεια του αντικειμένου. Η δυναμική του συστήματος αυτού, αναφερόμενη και ως ``κίνηση τσιμπήματος -- Pinching Motion" περιγράφεται με Lagrangian δυναμικό μοντέλο μόνο για τη κίνηση σε επίπεδο 2 διαστάσεων. Ο έλεγχος βασίζεται στο ότι το σύστημα είναι παθητικό (Passivity Based Control) και περιλαμβάνει τρεις προστιθέμενους επιμέρους όρους. Ο πρώτος όρος πραγματοποιεί ευσταθή δυναμική λαβή ελέγχοντας τις εσωτερικές δυνάμεις με τον δεύτερο να αναλαμβάνει τη περιστροφή του αντικειμένου, ενώ ο τρίτος όρος, όρος μάθησης, ανανεώνεται επαναληπτικά με τη χρήση διαφορικού όρου ολοκλήρωσης δυναμικής παλινδρόμησης (regression) ο οποίος βασίζεται στα δυναμικά χαρακτηριστικά του Lagrangian ρομποτικού μοντέλου του χεριού. Στόχος του ελέγχου αποτελεί η πραγματοποίηση σταθερής περιοδικής κίνησης με σταδιακή σύγκλιση στην επιθυμητή τροχιά με μηδενισμό του σφάλματος θέσης ως προς το αντικείμενο μέσω του όρου δυναμικής μάθησης. Η τεχνική αυτή γενικεύεται και σε δάκτυλα 3DOFs με επιτυχή αποτελέσματα~\cite{NguyenArimoto2002}.  \\

\textbf{Έλεγχος με εύκαμπτα σημεία επαφής}
\vspace{1ex}

Όπως ήδη έχουμε αναφέρει οι περισσότερες αναλυτικές μελέτες κάνουν υπόθεση άκαμπτων δακτύλων και άκαμπτου αντικειμένου περιβάλλοντος. 
Αυτό στις πρακτικές εφαρμογές δημιουργεί προβλήματα αστάθειας και προσαρμοστικότητας σε διαφορετικές καταστάσεις. Για το λόγο αυτό διαφαίνεται μία ιδιαίτερη ανάπτυξη του κλάδου των εύκαμπτων ρομποτικών συστημάτων (Soft Robotics).

Μία ενδιαφέρουσα τεχνική για το χειρισμό αντικειμένων με 2 εύκαμπτα ακροδάκτυλα τα οποία διαθέτουν προσαρμοσμένους αισθητήρες δύναμης κάνει χρήση κλασσικού PID ελεγκτή στις αρθρώσεις όπου η επιθυμητή θέση δρα ως ρυθμιστής της ασκούμενης δύναμης~\cite{Yoshikawa2010}. Πιο αναλυτικά αρχικά ορίζεται φίλτρο το οποίο αποθορυβοποιεί το σήμα από τους αισθητήρες δύναμης. Στη συνέχεια ορίζεται η δύναμη λαβής ως η μικρότερη τιμή από τις δυνάμεις οι οποίες δρουν στο αντικείμενο στον άξονα μεταξύ των δύο σημείων επαφής. Το σφάλμα μεταξύ της δύναμης λαβής και της επιθυμητής ορίζεται ως ανάλογο του ρυθμού μεταβολής της απόστασης των ακροδακτύλων από το κέντρο. Βάσει ανάστροφης κινηματικής ορίζεται τελικά η επιθυμητή θέση των αρθρώσεων.  Μία προέκταση που γίνεται στο συγκεκριμένο σχήμα είναι η εισαγωγή τεχνικής συμμόρφωσης για μεγάλες δυνάμεις. Στην περίπτωση αυτή όταν ανιχνευθεί εξωτερικά ασκούμενη δύναμη πάνω στο αντικείμενο μεγαλύτερη από ένα κατώφλι τότε το κέντρο μεταξύ των δύο σημείων επαφής μετακινείται προς τη κατεύθυνση της εξωτερικά ασκούμενης δύναμης με σκοπό να τη μειώσει. Αυτή η τεχνική μπορεί να χρησιμοποιηθεί για την εκτέλεση εργασιών όπου οι εξωτερικές δυνάμεις οδηγούν την εργασία~\cite{Yoshikawa2010}.

%\textbf{Χρήση Παρατηρητή για Απόριψη Διαταραχών}
%\vspace{1ex}


\subsection{Θέματα Ευστάθειας}

Μία από από τις σημαντικότερες ιδιότητες της λαβής είναι η ευστάθεια αυτής. Στη βιβλιογραφία ο όρος της ευστάθειας χρησιμοποιείται βάσει δύο τουλάχιστον εννοιών~\cite{Bicchi:2000fk} και προσδιορίζεται κυρίως στο επίπεδο του αντικειμένου~\cite{KumazakiSvininLuoOdashimaHosoe2002}. Η μία έχει να κάνει με την ευστάθεια κατά Lyapunov, και υποδεικνύει (ασυμπτωτική) ευστάθεια αν η δυναμική του συστήματος είναι τέτοια, όπου όταν το αντικείμενο μετατοπίζεται από τη θέση/προσανατολισμό αναφοράς, να μένει κοντά (και τελικά να επιστρέφει σε αυτήν). Ένας δεύτερος ορισμός, κατά Lagrange stability, αναφέρει ότι ένα συντηρητικό σύστημα θεωρείται ευσταθές αν βρίσκεται σε ένα αυστηρό τοπικό ελάχιστο της δυναμικής του ενέργειας. Ο δεύτερος ορισμός είναι και ο κυρίαρχος στην ανάλυση και τη μελέτη της ευστάθειας συστημάτων ρομποτικής λαβής. Στην ανάλυση της ευστάθειας πρέπει να προσμετρηθούν και οι συμμορφωτικές δυναμικές του συστήματος, όπως τυχόν φαινόμενα ελαστικότητας στα δάκτυλα. 

Η ανάλυση της ευστάθειας κατά Lagrange παρουσιάζει κάποια πρακτικά προβλήματα. Στη μηχανική η θέση ότι ένα σημείο ισορροπίας είναι ασταθές αν δεν είναι ελάχιστο σημείο δυναμικού δεν αποδεικνύεται και για συστήματα με περισσότερους από 2 βαθμούς ελευθερίας. Σε μία πραγματική εφαρμογή μπορούν επίσης να επιδρούν και μη συντηρητικές δυνάμεις, δημιουργούμενες από ατέλειες στη μηχανική δομή των στοιχείων, ή/και από το νόμο ελέγχου, καθιστώντας την ανάλυση της ευστάθειας κατά Lagrange μη έγκυρη. Η ανάλυση της ευστάθειας κατά Lyapunov όπως και άλλες δομικές ιδιότητες (ελεγξιμότητα, παρατηρησιμότητα, σταθεροποίηση) στο γενικό σύστημα της λαβής, αναλύονται ως προς τη γραμμική τους προσέγγιση όπου το σύστημα είναι πρακτικά ελαστικό γύρω από το σημείο ισορροπίας. 

Σε ένα ελαστικό σύστημα, ευστάθεια πρακτικά μεταφράζεται ως η ιδιότητα σύμφωνα με την οποία, οποιαδήποτε μετατόπιση από το σημείο ισορροπίας θα δημιουργήσει δύναμη η οποία θα τείνει να επιστρέψει το σύστημα στη κατάσταση ισορροπίας (\textit{stiffness--effect}). Ενεργειακά αυτό ερμηνεύεται ως προς το έργο που πρέπει να παράξει το σύστημα για την επαναφορά στο σημείο ισορροπίας. Για την ανάλυση αυτού του προβλήματος συνήθως δομείται αρχικά η σχέση μεταξύ των μητρών ακαμψίας μεταξύ αντικειμένου, ακροδακτύλων και αρθρώσεων η οποία προκύπτει από τη στατική ανάλυση του συστήματος στο επίπεδο του αντικειμένου μέσω της μήτρας λαβής και της Ιακωβιανής μήτρας του χεριού. Από αυτή τη σχέση εξάγονται οι συνθήκες ευστάθειας της λαβής συνήθως βάσει κάποιων απλουστευτικών παραδοχών προς διευκόλυνση της ανάλυσης όπως για παράδειγμα παραλείποντας τη βαρυτική επίδραση ή άλλες τυχόν μη μοντελοποιημένες επιδράσεις και όρους δεύτερης τάξης. Σε αυτή την απλουστευτική παραδοχή θέτοντας τη μήτρα ακαμψίας του αντικειμένου θετικά ορισμένη εξασφαλίζεται η ευστάθεια για το ελαστικό σύστημα. Συνήθως οι μήτρες ακαμψίας επιλέγονται διαγώνιες εξασφαλίζοντας ευκολία στην ανάλυση και αποσυζευγμένη συμπεριφορά ως προς τις καρτεσιανές διευθύνσεις. 

Προφανώς απαραίτητη κρίνεται η κατάλληλη αντιστάθμιση των διαφόρων επιδράσεων στο σύστημα για την επίτευξη ευστάθειας.  Πάνω σε αυτή τη κατεύθυνση στην ~\cite{ChoiChungYoum1994} προτείνεται αλγόριθμος κατάλληλης επιλογής μήτρας ακαμψίας για το αντικείμενο μέσω επαναληπτικής ευριστικής μεθόδου η οποία εγγυάται την ευστάθεια για το ελαστικό σύστημα λαμβάνοντας υπόψη κάθε φορά τη βαρυτική επίδραση καθώς και την επίδραση των εσωτερικών δυνάμεων. Στην ~\cite{KumazakiSvininLuoOdashimaHosoe2002} θεωρώντας ότι στο σύστημα της λαβής επιδρά νόμος αμέσου ελέγχου δύναμης (direct force control) στα ακροδάκτυλα (Closed Loop),  δομούνται οι συνθήκες ευστάθειας της λαβής για το γραμμικοποιημένο ελαστικό σύστημα με βαρυτική επίδραση, υπό απλό μοντέλο σημειακών επαφών με τριβή,επεκτείνοντας τις αντίστοιχες συνθήκες για το ανοικτό (Open Loop) σύστημα. Μία άλλη προσέγγιση προτείνει την εισαγωγή παρατηρητή διαταραχών (disturbance observer)στο σχήμα ελέγχου είτε του σερβομηχανισμού είτε στον έλεγχο δύναμης ή συμμόρφωσης, για την αντιστάθμιση της αβεβαιότητας, του βαρυτικού όρου, της ελαστικότητας και της τριβής των συνδέσμων. Η αποτελεσματικότητα και οι επιδόσεις του παρατηρητή διαταραχών κρίνεται υψηλή αλλά η ευστάθεια δεν μπορεί να καταστεί εγγυημένη βάσει αναλυτικής λύσης εκ των προτέρων πριν την εφαρμογή του ελεγκτή στο τελικό σύστημα. Οι συνθήκες ευστάθειας που δομούνται στις ερευνητικές προσπάθειες μέχρι στιγμής εξετάζονται μόνο ως προς ορισμένες συνιστώσες των διαταραχών στο σύστημα και οι παράμετροι του ελεγκτή ρυθμίζονται είτε μέσω επαναληπτικών ή γραφικών μεθόδων, είτε διαισθητικά~\cite{NakashimaYoshimastsuHayakawa2010}. Στην \cite{NakashimaYoshimastsuHayakawa2010} δομούνται αρχικά αναλυτικές ικανές συνθήκες για την εξασφάλιση ευστάθειας (stiffness--effect) για μικρές μετατοπίσεις γύρω από το σημείο ισορροπίας όπου υπάρχει ελαστική περιοχή λειτουργίας υπό την επίδραση του βαρυτικού όρου του αντικειμένου και ως προς τη μάζα αυτού, τα σημεία επαφής και τους ελαστικούς όρους των ακροδακτύλων. Στη συνέχεια για τη ρύθμιση των παραμέτρων του ελεγκτή (ο οποίος ενσωματώνει παραρατηρητή αντιστάθμισης διαταραχών--Disturbance Observer) για την εξασφάλιση ευστάθειας, δομείται συνάρτηση Lyapunov για το σύστημα από την οποία απαιτείται και η ικανοποίηση των αντίστοιχων συνθηκών ευστάθειας. Η ασυμπτωτική ευστάθεια αποδεικνύεται με την αρχή αμεταβλητότητας του LaSalle (LaSalle's invariance principle).
 


Για τη σύγκριση της ευστάθειας μεταξύ διαφορετικών λαβών αλλά και γενικότερα  για την ανάλυση της ευστάθειας κρίνεται χρήσιμος ο υπολογισμός μίας αντίστοιχης κατάλληλης μετρικής, όπως για παράδειγμα το πραγματικό μέρος των κυρίαρχων ιδιοτιμών του γραμμικοποιημένου μοντέλου λαβής. Μία πιο χρήσιμη προσέγγιση σε αρκετές εφαρμογές θα ήταν ο υπολογισμός της της διανυσματικής βάσης των γενικευμένων ελκτικών δυνάμεων γύρω από το σημείο ισορροπίας, παρέχοντας έτσι πληροφορία σχετικά με το εύρος του μέτρου των δυνάμεων για τις οποίες το σύστημα είναι ευσταθές. Αποδοτικοί αλγόριθμοι για τον υπολογισμό αυτού του μεγέθους δεν έχουν ακόμα αναπτυχθεί~\cite{Bicchi:2000fk}.


%Σημαντική είναι επίσης και η δουλειά που έχει γίνει στον έλεγχο συστημάτων λαβής υπό την παρουσία αβεβαιότητας ως προς κάποιες παραμέτρους, κατάσταση αρκετά συνήθης σε πραγματικές υλοποιήσεις.
%Είναι σημαντικό να αναφέρουμέ την απουσία ικανών γεωμετρικών συνθηκών για την εξασφάλιση ευστάθειας σε σύστημα λαβής πολών δακτύλων με έλεγχο συμμόρφωσης\cite{KimYiOhSuh2003}. 



%\section{Αρχιτεκτονικές Υλοποιήσης -- Hardware Implementation Architectures}


\end{document}
