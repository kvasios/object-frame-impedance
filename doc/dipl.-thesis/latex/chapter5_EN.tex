\documentclass[KVasios-ECE-Dipl.-Thesis-EN.tex]{subfiles}
\begin{document}
\chapter{Conclusions \& Future Work}

\noindent
In this final chapter, we summarize the main points of this diploma thesis, the conclusions that arise, as well as possible future directions and extensions.

\section{Summary}

This diploma thesis addresses a particularly demanding topic in the field of robotics, namely dexterous robotic manipulation using anthropomorphic robotic-hand configurations (with parallel cooperating kinematic chains).

First, an extensive literature review is presented on fundamental topics related to dexterous robotic manipulation, with an emphasis on the employed control techniques.

For the study and development of solutions at the control level of robotic grasping, we developed a custom simulation environment within Simulink, using the open-source Open Dynamics Engine libraries compiled into a MEX C++ S-Function block, with the goal of building a multibody-dynamics simulation platform with contact--collision handling.

On this platform, we initially apply an object-level dexterous-grasping control technique based on passivity, in combination with dynamic damping control at the object level. The results of these simulations demonstrate reliable execution of dynamic simulations based on the ODE API, always within the strict assumptions set from the outset, primarily regarding the physics of contacts between bodies. Based on these results, the particular algorithm is also evaluated, and although it exhibits overall satisfactory behavior, our critique focuses mainly on the following points,
\newpage
\begin{itemize}
\item
Large tangential pressure on cylindrical or cubic objects at the fingertips, resulting in possible loss of support due to motion outside the friction constraints of the contact model.
\item
Presence of steady-state final-position errors.
\item
Presence of parasitic coupling between motions in different directions.
\item
Error in the measured stiffness compared to the stiffness defined at the controller level.
\end{itemize}
We attribute these effects to the internal-force control technique through defining stiffness coefficients between the fingertips and a virtual object frame. The forces generated by these connections are not mapped into the null space of the grasp matrix, resulting in a resultant external force on the manipulated object, which is also a cause of the aforementioned problems.

On this issue, we propose a solution based on the geometry of the object surface, and specifically the projection (into the internal subspace of the grasp matrix) of forces normal to the contact surface. We consider this to be a realistic approach, since acquiring the surface-normal direction is possible through vision--touch sensor suites which, in most cases, are present in robotic applications targeting dexterous manipulation. In this way, we keep fingertip forces within the friction constraints of the contact model, avoiding possible slipping, while simultaneously achieving explicit control of the internal forces. The performed simulations verify exactly these hypotheses, yielding smaller steady-state position errors, smaller couplings between motions in different directions, as well as a consistent stiffness matrix compared to the one defined.

Finally, we introduce the gravitational effect into the system, which, as an external force, creates steady-state position errors. For the fingers we have already introduced a dynamic gravity term, but not for the object. By extending the stiffnesses with respect to translation--rotation, we introduce an object-level gravity-compensation term, geometrically consistent, with successful results.\\

\newpage
\section{Extensions -- Future Directions}

\textbf{Simulation}
\\

The simulation of dynamic systems constitutes an extremely large and critical part of systems engineering for building complete, successful applications. As already mentioned, Open Dynamics Engine is a particularly popular solution for simulating robotic systems, mainly due to its robustness, stability, flexibility, and speed. Nevertheless, it exhibits significant limitations and simplifications, while at the same time it is restricted to simulations of perfectly rigid bodies. In contrast, a significant part of robotics research is oriented toward soft--flexible robotic components (Soft Robotics). Although for such cases one can seek customized commercial solutions in industry, the problem of realistic multibody dynamic simulation with simultaneous collision handling remains open.

Therefore, today one could develop a new open-source software package for dynamic simulation, tailored to the current needs of the robotics community. It is interesting that, although there have been numerous efforts to create robotic development platforms (Webots, V-REP, ROS, etc.), they are all based on simplified physics engines oriented primarily toward entertainment and graphics applications, with the main ones being ODE, Bullet, and PhysX, where the objective is fast and plausible response rather than dynamic simulation with absolute accuracy of results.\\


\textbf{Control}
\\

Regarding control, one can first observe that in our system we assume a perfect joint-torque controller, where our command signal translates perfectly into the torque ultimately applied at the joints. This naturally removes a significant portion of the simulation's reliability. Therefore, an obvious extension would be the creation of a low-level torque-control subsystem that includes the dynamics of the actuation system.

Regarding the overall dynamic system of the robotic-grasp model, as already mentioned, in our analysis we made certain simplifying assumptions to facilitate the determination of the dynamic damping term. This results in a non-negligible inconsistency in the character of the transient characteristics for different configurations of the system. It would be useful to investigate, in this direction, a more complete control scheme that would also take into account the Coriolis terms (products of angular velocities) that we omitted in our analysis.

Subsequently, what becomes evident is that in our algorithm we assume the inertial characteristics of the manipulated object are known, which undermines the effort to build robotic systems capable of automated action in unstructured environments. It is possible to introduce an adaptive-control scheme that can tune the parameters of the system dynamically for different object inertial characteristics.

Although the system satisfies the characterization ``dexterous'', in practice there is significant room for improvement, considering that the object's orientation workspace is in fact limited. A particularly interesting prospect would therefore be an extension of the scheme to perform finger gaiting and/or suitable sliding--rolling at the fingertips, to increase dexterity, essentially introducing hybrid/discrete-event dynamics in this way.

Another line of work with substantial interest would be testing these algorithms on systems with different contact models that also include second-order effects.\\


\textbf{Real System}
\\

Theoretical analysis and simulation are inseparable and necessary parts of any effort to develop automatic-control systems. However, all of this has limited meaning if there is no application in the real world and under real conditions.

\end{document}


